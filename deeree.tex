\documentclass[a4paper, oneside]{book}
\usepackage{jipkg}
\usepackage{multirow}

%few ipa symbols
\newcommand{\ipaE}{ε}
\newcommand{\ipaET}{ɛ̃}
\newcommand{\ipaO}{ɔ}
\newcommand{\ipaAT}{ɑ̃}
\newcommand{\ipaS}{ʃ}
\newcommand{\ipaZ}{ʒ}
\newcommand{\ipaR}{ʁ}

%glossary
\usepackage[toc]{glossaries}
\newcommand{\newgle}[2]{\newglossaryentry{#1}{name=#1, description={#2}}}
\glossarystyle{listgroup}
\makeglossaries

%gloss and examples
\usepackage{gb4e}

\title{Deẽreẽ\\Grammar of a fantasy constructed language}
\author{Jirsad \textsc{Ïoh}}
\date{2016 — \today}
\jihypersetup{Deẽreẽ — A fantasy conlang}{Jirsad ÏOH}

\begin{document}
\selectlanguage{english}
\frontmatter
\maketitle
\tableofcontents
\jichap{Introduction}
\begin{exe}
\ex\label{ex:intro-cat}
\gll mosoj’a fãssa eat a\\
cat-3S wizard 3S.INDEF 3S\\
\trans ‘The cat is a wizard’
\end{exe}

In example \ref{ex:intro-cat}, we see that blah blah blah…
\todo{fig:glossing abbreviations}

\mainmatter
%-----
\jichap{Phonology}\label{chap:phonology}
Phonology regards the phonemes incorporated into the language, consonants or vowels, and their use
in syllables and words.

\jisec{Consonants}

\begin{table}[h]\label{tab:phon-ipa-cons}
\begin{center}
\begin{tabular}{|r|c|c|c|c|c|c|c|}
\hline
Manner/Point of articulation & Bilabial & Labiodental & Alveolar & Postalveolar & Palatal & Velar & Uvular\\\hline
Nasal			&   m &     &   n &     &     &     &     \\\hline
Plosive			& p b &     & t d &     &     & k g &     \\\hline
Fricative		&     & f v & s z & \ipaS \ipaZ & &&\ipaR \\\hline
Approximant		&     &     &     &     &   j &     &     \\\hline
Lateral approximant	&     &     &   l &     &     &     &     \\\hline
\end{tabular}
\end{center}
\caption{Consonants in Deẽreẽ}
\end{table}

Romanization is done with the IPA symbols, that is, the ones in table \ref{tab:phon-ipa-cons},
except for \ipaS, \ipaZ~and \ipaR, which are written respectively as \emph{sh}, \emph{zh} and
\emph{r}. Pronunciation of these three letters is:

\begin{description}
\item[sh] as in English ‘\emph{sh}eep’
\item[zh] as in English ‘vi\emph{si}on’
\item[r]  as in French  ‘F\emph{r}ance’
\end{description}

\jissec{Consonant pairs}
The combinations of consonants allowed are the following: \textbf{tr, dr, pl, bl, kl, gl, kr, gr}.
Other will occur due to phonotactics, but no other beginning in a plosive.

\jisec{Vowels}
The nine vowels are as described (IPA and orthography) in table \ref{tab:phon-ipa-vowels}.

\begin{table}[h]\label{tab:phon-ipa-vowels}
\begin{center}
\begin{tabular}{|r|c|c|c|c|}
\hline
	& Front & Back \\\hline
Closed	& /i/, /y/ <ù> & /u/ \\\hline
Mid	& /e\textasciitilde\ipaE/, /\ipaET(j)/ <ẽi>, /ø{\textasciitilde}œ/ <ë> & /o\textasciitilde\ipaO/ \\\hline
Open	& /a/ & /\ipaAT/ <ã> \\\hline
\end{tabular}
\end{center}
\caption{Vowels in Deẽreẽ}
\end{table}

\jissec{On /{\ipaET}j/ or /\ipaET/}
One of the doubtful phonemes in Deẽreẽ’s phonology is the grapheme <ẽi>’s pronunciation. It can be
pronounced with palatalization or not, that is, /{\ipaET}j/ or /\ipaET/. How to pronounce it is
normally determined by its position in the word.

As the last phoneme of a word <ẽi> will be palatalized: /{\ipaET}j/. But when <ẽi> is initial or
inside a word, it will be pronounced as either palatalized or not. When followed by j, l, s, n, p,
t, or sh, that is a consonant from bilabial to palatal but not further back (k, g, r), it is
palatalized, otherwise it is not.

\jisec{Stress}
Stress in words is basically on the last syllable.
\begin{itemize}
\item zhalbar\emph{ki} (beer)
\item loj\emph{shë} (history)
\end{itemize}

\newgle{ojmata}{to predict}
\newgle{tëjete}{to forget}
However, stress on infinitive verbs is on the second-to-last syllable, for example with
\gls{tëjete}, \gls{drete} and \gls{ojmata}.
\begin{itemize}
\item të\emph{je}te (to forget)
\item \emph{dre}te (to go)
\item oj\emph{ma}ta (to predict)
\end{itemize}

\newgle{forfof}{desert}
\newgle{aù}{little/small}
And a when a word takes a person suffix, this does not change stress location. Examples with
\gls{forfof} and \gls{aù}.
\begin{itemize}
\item for\emph{fof}’a (the desert)
\item a\emph{ùr}’mer (these little ones)
\end{itemize}

\jisec{Phonotactics}
The syllable structure is \textbf{(C|B)V(L)(F)} where:
\begin{description}
\item[C] is any consonant
\item[B] is a group of two consonants among \emph{tr, dr, pl, bl, kl, gl, kr, gr}
\item[V] is any vowel
\item[L] is a consonant among \emph{j, l, r, s}
\item[F] word-final only, a consonant among \emph{n, p, t, k, f, s, sh, l}
\end{description}

\jissec{Vowel pairs}
Quite often, two vowels will appear in a word consecutively. In this case, there are three cases:
long vowel, diaeresis, or semi-vowel.

\jisssec{Long vowel sounds}
In a word like \newgle{aal}{wind}\gls{aal}, There is a long vowel sound due to vowel reduplication.
There, it is pronounced as [a.al], instead of [a:l]. That is, the real pronounciation of this word
is as close to diaeresis as possible, as opposed to a ‘real’ long vowel.

\jisssec{Diaeresis}
\newgle{tëfuete}{to wipe}
\newgle{fùã}{powder}
Two consecutive vowels will give a diaeresis, that is, the two vowels being pronounced separately
without any semi-vowel, in words like \gls{fùã}, powder, or before a verb ending, for example in
\gls{tëfuete}, to wipe.

\jisssec{Semi-vowel}
\newgle{marùëlf}{nature}
A semi-vowel, that is, a sound like /j/, /w/, or /ɥ/ (the \emph{u} in the French huit), is used in
various words, for example the Deẽreẽ word for nature, \gls{marùëlf}.



%-----
\jichap{Morphology}\label{chap:morphology}
This chapter, morphology, aims to describe \emph{what the words look like}. This means describing
the nouns, adjectives, adverbs, verbs in their basic forms. Details though will be about person
suffixes, Deẽreẽ’s equivalent for articles.

\jisec{Gender}
Deẽreẽ has three genders: Human, marked \emph{H}, Magic, marked \emph{M}, and Common, marked
\emph{C}. Every word has a gender, whatever its grammatical nature (part of speech) might be: nouns
have gender, but verbs have gender too, as well as adjectives. Words that are not nouns do not just
agree in gender (if they do), but has their own intrinsic categories.

To determine the gender of a word, you have to look at its last vowel (whether or not it ends in a
vowel). The rules are in table \ref{tab:morph-genders}.

\begin{table}[h]\label{tab:morph-genders}
\begin{center}
\begin{tabular}{|c|c|c|c|}
\hline
\textbf{Human}  & e & ẽi & u\\\hline
\textbf{Magic}  & a & ã  & o\\\hline
\textbf{Common} & ù & ë  & i\\\hline
\end{tabular}
\end{center}
\caption{Vowels in the Deẽreẽ genders (H:e/ẽi/u, M:a/ã/o, C:ù/ë/i)}
\end{table}

For example:
\begin{description}
\newgle{zhalbarki}{beer}\newgle{drete}{to go}\newgle{magra}{soul}
\item[\gls{zhalbarki}] beer, ending in \emph{i}, is of \emph{Common} gender
\item[\gls{drete}] to go, ending in \emph{e}, is of \emph{Human} gender
\item[\gls{magra}] soul, ending in \emph{a}, is of \emph{Magic} gender
\end{description}


\jisec{Number}
Number in Deẽreẽ is simply singular (\emph{SI}) and plural (\emph{PL}). As gender is marked with the
last vowel of a given word, number is marked with the last consonant.

\jissec{Formation}
\begin{table}[h]\label{tab:morph-number}
\begin{center}
\begin{tabular}{|c|ccccccccc|}
\hline
\textbf{Singular} & -l & -n & -p & -t & -k & -f & -s & -sh & -\emph{vowel}\\\hline
\textbf{Plural}   & -r & -m & -b & -d & -g & -v & -z & -zh & -r\\\hline
\end{tabular}
\end{center}
\caption{Number (singular and plural) formation}
\end{table}

For example, magra, magrar; aal; aar; \newgle{sùn}{metal}\gls{sùn} (metal), sùm, \newgle{zërsh}{pig}
\gls{zërsh} (pig), zërzh.

\jissec{Meaning and agreement}
\jisssec{Nouns}
The meanings of singular and plural for nouns are as expected: singular means that the thing is
present once, plural that it is present several times: \newgle{zheo}{fairy}\gls{zheo}, fairy,
\gls{zheo}r, fairies.

As for uncountable nouns, singular means ‘some stuff’, while plural means ‘several kinds of the
stuff’. This also is quite an expected behavior: \newgle{lojshë}{history}\gls{lojshë}, history,
some part of history; \gls{lojshë}r, histories, or \newgle{geni}{milk}\gls{geni}, milk, some milk;
\gls{geni}r, several kinds of milk.

\jissec{Verbs}
Verbs agree in number with the agent. See chapter Conjugation on page \pageref{chap:conjugation} for
more detail.

\jissec{Adverbs and adjectives}
These do not agree, as is explained in chapter Syntax (page \pageref{chap:syntax}).


\jisec{Person suffixes}
Person suffixes in Deẽreẽ are suffixes that can be added to different kinds of words, notably nouns,
verbs, adjectives, adverbs, or postpositions. Their purpose is to precise which person the word
refers to. The language uses six grammatical persons, the usual ones:

\begin{description}
\item[1.SI] I, speaker
\item[2.SI] singular you (thou), whom is spoken to
\item[3.SI] he/she/it, neither speaker nor person being spoken to
\item[1.PL] we, several people including the speaker
\item[2.PL] plural you, several people including the one(s) spoken to
\item[3.PL] they, several people neither speaking nor being spoken to
\end{description}

\jissec{Form and agreement}
Person suffixes have to agree with what they \emph{refer to}. This is very important because, the
word they refer to is not always the word on which they are. They decline in \emph{number},
\emph{person}, and \emph{gender}. Their basic forms, preceded by an apostrophe, are listed in table
\ref{tab:morph-basic-pers-suff}.

\begin{table}[h]
\begin{center}\label{tab:morph-basic-pers-suff}
\begin{tabular}{|c||c|c|c|c|}
\hline
\multirow{2}{*}{\textbf{Gender}} & \multicolumn{2}{|c|}{\textbf{Human}} & \multirow{2}{*}{\textbf{Magic}} & \multirow{2}{*}{\textbf{Common}}\\
                                 & \textbf{Female/neuter} & \textbf{Male} & & \\\hline\hline
\textbf{1.SI}                    & ’esh & ’ẽish & ’ash & ’ùsh \\\hline
\textbf{2.SI}                    & ’el  & ’ẽil  & ’al  & ’ùl  \\\hline
\textbf{3.SI}                    & ’e   & ’ẽi   & ’a   & ’ù   \\\hline
\textbf{1.PL}                    & ’ezh & ’ẽizh & ’azh & ’ùzh \\\hline
\textbf{2.PL}                    & ’er  & ’ẽir  & ’ar  & ’ùr  \\\hline
\textbf{3.PL}                    & ’er  & ’ẽir  & ’ar  & ’ùr  \\\hline
\end{tabular}
\end{center}
\caption{Person suffixes in their basic form}
\end{table}

We have to notice there is no difference between \textbf{2.PL} and \textbf{3.PL}. The difference is
thus done with context understanding.

\jissec{Definite}
The definite article, that is, the equivalent of \emph{the}, is formed simply by applying the person
suffix after the noun, in their basic form from table \ref{tab:morph-basic-pers-suff}. For example
see (\ref{exe:person-def}), with \newgle{mosoj}{cat}\gls{mosoj}, cat.

\begin{exe}
\ex\label{exe:person-def}
\begin{xlist}
\ex\gll mosoj’a\\
cat(MAG)-3.SI.MAG\\
\trans ‘the cat’

\ex\gll mosoj’ash\\
cat(MAG)-1.SI.MAG\\
\trans ‘I, the cat’

\ex\gll mosojr’ar\\
cat(MAG)-2.PL.MAG\\
\trans ‘you, cats’
\end{xlist}
\end{exe}

As we see in this example (\ref{exe:person-def}), the person suffixes add meaning to the words in a
short, efficient way. They are also used with verbs, as in \emph{dret’ẽil}, I go. Details on
conjugation are however in a separated chapter.

\jissec{Indefinite}
Indefinite articles express the idea that the speaker doesn’t know \emph{which} thing they are
speaking about, but \emph{a} thing. Its form in Deẽreẽ is a separate word, not a suffix. This word
is \newgle{et}{indefinite article}\emph{et}, which declines as described in table \ref{tab:morph-indef-pers-suff}.

\begin{table}[h]
\begin{center}\label{tab:morph-indef-pers-suff}
\begin{tabular}{|c||c|c|c|c|}
\hline
\multirow{2}{*}{\textbf{Gender}} & \multicolumn{2}{|c|}{\textbf{Human}} & \multirow{2}{*}{\textbf{Magic}} & \multirow{2}{*}{\textbf{Common}}\\
                                 & \textbf{Female/neuter} & \textbf{Male} & & \\\hline\hline
\textbf{1.SI}                    & etesh & etẽish & et’ash & et’ùsh \\\hline
\textbf{2.SI}                    & etel  & etẽil  & et’al  & et’ùl  \\\hline
\textbf{3.SI}                    & et    & ẽit    & eat    & eùt    \\\hline
\textbf{1.PL}                    & etezh & etẽizh & et’azh & et’ùzh \\\hline
\textbf{2.PL}                    & eter  & etẽir  & et’ar  & et’ùr  \\\hline
\textbf{3.PL}                    & ed    & ẽid    & ead    & eùd    \\\hline
\end{tabular}
\end{center}
\caption{Indefinite article \emph{et} and its declensions}
\end{table}

The position of the indefinite article is still after the noun it describes. See example
\ref{exe:person-indef}.

\begin{exe}
\ex\label{exe:person-indef}
\begin{xlist}
\ex\gll \gls{eltol} eat\\
bird(MAG) 3.SI.MAG.INDEF\\
\trans ‘a bird’

\ex\gll \newgle{shil}{worm}\gls{shil} et’ùl\\
worm(COM) 2.SI.COM.INDEF\\
\trans ‘You worm!’
\end{xlist}
\end{exe}

\jissec{Demonstrative}
Without a pretty table this time; demonstrative articles are used to designate a specific object out
of a group, as a transition from indefinite to definite. The English equivalent are \emph{this} and
\emph{that}.

These are of two types: proximal and distal.

\jisssec{Proximal demonstrative}
The word proximal means \emph{close}, so a proximal demonstrative is an article designating a thing
close to the speaker (\emph{this}). In Deẽreẽ, it is formed with the definite suffixes preceded with
the letter \emph{m}.

\begin{exe}
\ex\label{exe:person-prox-dem}
\gll mosoj’ma\\
cat(MAG)-3.SI.PDEM\\
\trans this cat
\end{exe}

\jisssec{Distal demonstrative}
As proximal means close, distal means \emph{remote}. It is so the equivalent of \emph{that}. It is
made in the language with the definite suffixes preceded with \emph{asm}. It is the occasion here to
introduce the word \newgle{senù}{fish}\gls{senù}, fish.

\begin{exe}
\ex\label{exe:person-dist-dem}
\gll \gls{senù}’asmù\\
fish(COM)-3.SI.PDEM\\
\trans that fish
\end{exe}

\jissec{Genitive}
’re
\jisssec{Owner}
todel’rẽi
\jisssec{Distinct}
ëj adal’e ãblẽi’re



\jisec{Adjective}
\todo{before: all, no suffix}

\todo{after: one only, 3S suffix}
\jisec{Adpositions}
\todo{post-positions a lot}

\todo{pre-positions with stuff.}

%-----
\jichap{Derivational Morphology}\label{chap:derivational-morphology}
The name ‘derivational morphology’ refers to all the mechanisms of word derivation, that is, how to
make a word out of another. These rules apply to several kinds of words, and are hereafter grouped
by part of speech they derive.

\jisec{Noun derivations}
\jisec{Verb derivations}
\todo{1t1 --> 2lf: tool, that with which X is done\\trasata: to give --> trasãlf: generosity}

\todo{1t1 --> 31n: result, that which X does\\torete: to say --> toruen: speech}

%-----
\jichap{Conjugation}\label{chap:conjugation}
\jisec{Tense}
\jissec{Present}
\todo{et}
\jissec{Past}
\todo{ẽiet}
\jissec{Future}
\todo{et trã}
\jisec{Mood}
\jissec{Indicative}
\todo{et}
\jissec{Participle}
\todo{ej}
\jissec{Interrogative}
\todo{Vf es}
\jissec{Subjunctive}
\todo{Vf es}
\jissec{Imperative}
\todo{Vf a-V-et}
\jisec{Aspect}
\jissec{Perfect/Imperfect}
\todo{drej}
\jissec{Cessative/Inchoative}
\todo{drẽiej/drej trã}
\jisec{The Copula}
\todo{first suffix repetition}

Mosoj’a fãssa eat’a. => the cat is a wizard.

%-----
\jichap{Syntax}\label{chap:syntax}
\jisec{Noun phrase}
\jisec{Sentence}
\jissec{Basic order}
\jissec{Questions}
\jissec{Imperatives}
\jisec{Subclauses}
\todo{Vf}
\jisec{Existentials}
\todo{at}

%-----
\jichap{Pragmatics}\label{chap:pragmatics}
\jisec{Sayings}
\jisec{Samples}
\jissec{Ëj un naklër’ùr dësish’rù — The King of All Snakes}
\label{samp:theKingOfAllSnakes}
\emph{This was the first ‘poem’ written in Deẽreẽ.}

Asaber ëj edan’a mete’re,\\
Akùt’er i noshi nushata lavëdã’ar,\\
Atraùt’er ëj lëlù aal eùt ëj eltol’ra nën’rù,\\
Asãkrer ëj adal tis’mù mik’rù ak ãdijl’rar.

Iszher kerbo pitrëk’ù ùsaat ëj un naklër’ùr dësish’rù.\\
Sabeser un’me, kùs’er un’me, traùs’er un’me, sãkreser un’me, osket dësish’ù.\\
Dësistis’rù kla drẽiej ãgret kiden’me.\\
Pa vimẽilf’re nãjlat un grebas eltol’mar.

\jisssec{English translation}
Hear the river’s laugh,\\
See in the sky the clouds swim,\\
Know the calm of the great wind bird,\\
Love the riches and flowers of this old place.

In the yellow mangrove lives the king of all snakes.\\
Everything you hear, everything you see, everything you know, everything you love, belongs to the king.\\
This noble being hasn’t finished ruling his kingdom.\\
Every bird that flies sings his just name.


\newgle{naklë}{snake}

%-----
\backmatter
%\newgle{’esh}{je}
\newgle{’el}{tu}
\newgle{’e}{il}
\newgle{’ezh}{nous}
\newgle{’er}{vous}
\newgle{’er}{ils}
\newgle{’me}{ceci, celui-ci}
\newgle{’asme}{cela, celui-là}
\newgle{tis’mù}{ici}
\newgle{tis’asmù}{là}
\newgle{guẽi}{qui}
\newgle{guklë}{quoi}
\newgle{gutis}{où}
\newgle{guimof}{quand}
\newgle{gurshep}{comment}
\newgle{alk}{ne … pas}
\newgle{un}{tout}
\newgle{zhëmin}{beaucoup}
\newgle{oko}{quelques}
\newgle{ãjne}{peu}
\newgle{lij}{autre}
\newgle{??}{un}
\newgle{??}{deux}
\newgle{??}{trois}
\newgle{??}{quatre}
\newgle{??}{cinq}
\newgle{lëlù}{grand}
\newgle{tels}{long (espace)}
\newgle{vùãp}{long (temps)}
\newgle{dreri}{large}
\newgle{blujt}{épais}
\newgle{tùlzhùp}{lourd}
\newgle{aù}{petit}
\newgle{rosup}{court (espace)}
\newgle{ëkruzol}{court (temps)}
\newgle{grãsu}{étroit}
\newgle{sù}{mince}
\newgle{azhe}{femme}
\newgle{shùkẽi}{homme (mâle adulte)}
\newgle{ẽi}{homme (être humain)}
\newgle{nadës}{enfant}
\newgle{kësùj azhe}{femme (épouse)}
\newgle{kësùj shùkẽi}{mari}
\newgle{todel}{mère}
\newgle{ẽiuk}{père}
\newgle{borùk}{animal}
\newgle{senù}{poisson}
\newgle{eltol}{oiseau}
\newgle{pleù}{chien}
\newgle{unun}{pou}
\newgle{naklë}{serpent}
\newgle{shil}{ver}
\newgle{sãs}{arbre}
\newgle{milzhor}{forêt}
\newgle{pùk}{bâton}
\newgle{agor}{fruit}
\newgle{agubën}{graine}
\newgle{kij}{feuille (d'un végétal)}
\newgle{agatùs}{racine}
\newgle{lakil}{écorce}
\newgle{ãdijl}{fleur}
\newgle{blëjli}{herbe}
\newgle{lakẽil}{corde}
\newgle{mël}{peau}
\newgle{kif}{viande}
\newgle{zhok}{sang}
\newgle{ãrshisp}{os}
\newgle{faltùs}{graisse}
\newgle{fẽilkùr}{œuf}
\newgle{dokrish}{corne}
\newgle{umijsh}{queue (d'un animal)}
\newgle{dëglisnol}{plume (d'un oiseau)}
\newgle{drarri}{cheveux}
\newgle{urvẽip}{tête}
\newgle{klirëp}{oreille}
\newgle{ùkrùka}{œil}
\newgle{suzù}{nez}
\newgle{bẽi}{bouche}
\newgle{zhë}{dent}
\newgle{sëjsh}{langue (organe)}
\newgle{zhedrẽi}{ongle}
\newgle{saszãlf}{pied}
\newgle{ẽidù}{jambe}
\newgle{sebokẽi}{genou}
\newgle{tisgak}{main}
\newgle{lasãf}{aile}
\newgle{uplël}{ventre}
\newgle{sialash}{entrailles, intestins}
\newgle{urlil}{cou}
\newgle{ultrëlãs}{dos}
\newgle{dokap}{poitrine}
\newgle{shazha}{cœur (organe)}
\newgle{gëgrip}{foie}
\newgle{fete}{boire}
\newgle{krùtù}{manger}
\newgle{blasùtù}{mordre}
\newgle{giùtù}{sucer}
\newgle{reùtù}{cracher}
\newgle{ùglùtù}{vomir}
\newgle{fissata}{souffler}
\newgle{boata}{respirer}
\newgle{mete}{rire}
\newgle{kùtù}{voir}
\newgle{raldùtù}{entendre}
\newgle{traùtù}{savoir}
\newgle{sojmata}{penser}
\newgle{zhetiùtù}{sentir (odorat)}
\newgle{garata}{craindre}
\newgle{ëvata}{dormir}
\newgle{ùsaata}{vivre}
\newgle{nibete}{mourir}
\newgle{ëoata}{tuer}
\newgle{glasfete}{se battre}
\newgle{marùtù}{chasser (le gibier)}
\newgle{blifùtù}{frapper}
\newgle{drùdrete}{couper}
\newgle{dujvolata}{fendre}
\newgle{lùmùete}{poignarder}
\newgle{kroùtù}{gratter}
\newgle{plakùtù}{creuser}
\newgle{nushata}{nager}
\newgle{grebata}{voler (dans l'air)}
\newgle{blaskete}{marcher}
\newgle{rãùtù}{venir}
\newgle{vagrete}{s'étendre, être étendu}
\newgle{mirrete}{s'asseoir, être assis}
\newgle{saszata}{se lever, se tenir debout}
\newgle{gulata}{tourner (intransitif)}
\newgle{fuzeata}{tomber}
\newgle{trasata}{donner}
\newgle{mikrete}{tenir}
\newgle{plẽiokata}{serrer, presser}
\newgle{tëfuete}{frotter}
\newgle{zafeata}{laver}
\newgle{popiete}{essuyer}
\newgle{fùlpete}{tirer}
\newgle{zhofozhùtù}{pousser}
\newgle{piskete}{jeter, lancer}
\newgle{uzhëete}{lier}
\newgle{shãsete}{coudre}
\newgle{basete}{compter}
\newgle{nëdrete}{dire}
\newgle{nãjlata}{chanter}
\newgle{shùtù}{jouer (s'amuser)}
\newgle{fuzata}{flotter}
\newgle{fashùtù}{couler (liquide)}
\newgle{zùkata}{geler}
\newgle{blursùtù}{gonfler (intransitif)}
\newgle{esërof}{soleil}
\newgle{nosh}{lune}
\newgle{eleãt}{étoile}
\newgle{zhas}{eau}
\newgle{ëtã}{pluie}
\newgle{edan}{rivière}
\newgle{samas}{lac}
\newgle{ãsta}{mer}
\newgle{trãpãl}{sel}
\newgle{iros}{pierre}
\newgle{lëshall}{sable}
\newgle{tranùk}{poussière}
\newgle{ùrùk}{terre (sol)}
\newgle{lavëdã}{nuage}
\newgle{shugorlo}{brouillard}
\newgle{noshi}{ciel}
\newgle{aãtrël}{vent}
\newgle{grear}{neige}
\newgle{shop}{glace}
\newgle{omajt}{fumée}
\newgle{ijs}{feu}
\newgle{iklëjsh}{cendre}
\newgle{shãrijata}{brûler (intransitif)}
\newgle{tef}{route}
\newgle{tor}{montagne}
\newgle{lif}{rouge}
\newgle{suifi}{vert}
\newgle{kerbo}{jaune}
\newgle{ùlëdan}{blanc}
\newgle{grëjzã}{noir}
\newgle{tãi}{nuit}
\newgle{vash}{jour}
\newgle{fùjlëlãdal}{an, année}
\newgle{ãk}{chaud (température)}
\newgle{deer}{froid (température)}
\newgle{tëpolo}{plein}
\newgle{murzhẽi}{nouveau}
\newgle{adal}{vieux}
\newgle{asluk}{bon}
\newgle{gazorn}{mauvais}
\newgle{krãrvë}{pourri}
\newgle{gëjzes}{sale}
\newgle{ẽikrur}{droit (rectiligne)}
\newgle{vãtro}{rond}
\newgle{vaadip}{tranchant}
\newgle{eklinushal}{émoussé}
\newgle{limabo}{lisse}
\newgle{ruglẽies}{mouillé, humide}
\newgle{eupa}{sec}
\newgle{pa}{juste, correct}
\newgle{mato}{près}
\newgle{masëto}{loin}
\newgle{vëpi}{droite}
\newgle{virn}{gauche}
\newgle{i}{à}
\newgle{iszher}{dans}
\newgle{lùsh}{avec (ensemble)}
\newgle{ak}{et}
\newgle{krẽi}{si (condition)}
\newgle{ãbin}{parce que}
\newgle{vimẽilf}{nom}


\clearpage\glsaddall
\begin{multicols}{2}\printglossaries\end{multicols}
\listoffigures\addcontentsline{toc}{chapter}{\listfigurename}
\listoftables\addcontentsline{toc}{chapter}{\listtablename}
\listoftodos
\end{document}

\documentclass[a4paper, oneside]{book}
\usepackage{jipkg}
\usepackage{multirow}

%few ipa symbols
\newcommand{\ipaE}{ε}
\newcommand{\ipaET}{ɛ̃}
\newcommand{\ipaO}{ɔ}
\newcommand{\ipaAT}{ɑ̃}
\newcommand{\ipaS}{ʃ}
\newcommand{\ipaZ}{ʒ}
\newcommand{\ipaR}{ʁ}

%glossary
\usepackage[toc]{glossaries}
\newcommand{\newgle}[2]{\newglossaryentry{#1}{name=#1, description={#2}}}
\glossarystyle{listgroup}
\makeglossaries

%gloss and examples
\usepackage{gb4e}

\title{Deẽreẽ\\Grammar of a fantasy constructed language}
\author{Jirsad \textsc{Ïoh}}
\date{2016 — \today}
\jihypersetup{Deẽreẽ — A fantasy conlang}{Jirsad ÏOH}

\begin{document}
\selectlanguage{english}
\frontmatter
\maketitle
\tableofcontents
\jichap{Introduction}
\begin{exe}
\ex\label{ex:intro-cat}
\textbf{Fãssa mosoi’a.}\\
/'fã.s:a mo.'soi.a/

\gll fãssa mosoi-’a\\
wizard cat-3S\\
\trans ‘The cat is a wizard’
\end{exe}

In example \ref{ex:intro-cat}, we see that blah blah blah…
\todo{fig:glossing abbreviations}

\mainmatter
%-----
\jichap{Phonology}\label{chap:phonology}
Phonology regards the phonemes incorporated into the language, consonants or vowels, and their use
in syllables and words.

\jisec{Consonants}

\begin{table}[h]\label{tab:phon-ipa-cons}
\begin{center}
\begin{tabular}{|r|c|c|c|c|c|c|c|}
\hline
Manner/Point of articulation & Bilabial & Labiodental & Alveolar & Postalveolar & Palatal & Velar & Uvular\\\hline
Nasal			&   m &     &   n &     &     &     &     \\\hline
Plosive			& p b &     & t d &     &     & k g &     \\\hline
Fricative		&     & f v & s z & \ipaS \ipaZ & &&\ipaR \\\hline
Approximant		&     &     &     &     &   j &     &     \\\hline
Lateral approximant	&     &     &   l &     &     &     &     \\\hline
\end{tabular}
\end{center}
\caption{Consonants in Deẽreẽ}
\end{table}

Romanization is done with the IPA symbols, that is, the ones in table \ref{tab:phon-ipa-cons},
except for \ipaS, \ipaZ~and \ipaR, which are written respectively as \emph{sh}, \emph{zh} and
\emph{r}. Pronunciation of these three letters is:

\begin{description}
\item[sh] as in English ‘\emph{sh}eep’
\item[zh] as in English ‘vi\emph{si}on’
\item[r]  as in French  ‘F\emph{r}ance’
\end{description}

\jissec{Consonant pairs}
The combinations of consonants allowed are the following: \textbf{tr, dr, pl, bl, kl, gl, kr, gr}.
Other will occur due to phonotactics, but no other beginning in a plosive.

\jisec{Vowels}
The nine vowels are as described (IPA and orthography) in table \ref{tab:phon-ipa-vowels}.

\begin{table}[h]\label{tab:phon-ipa-vowels}
\begin{center}
\begin{tabular}{|r|c|c|c|c|}
\hline
	& Front & Back \\\hline
Closed	& /i/, /y/ <ù> & /u/ \\\hline
Mid	& /e\textasciitilde\ipaE/, /\ipaET(j)/ <ẽi>, /ø{\textasciitilde}œ/ <ë> & /o\textasciitilde\ipaO/ \\\hline
Open	& /a/ & /\ipaAT/ <ã> \\\hline
\end{tabular}
\end{center}
\caption{Vowels in Deẽreẽ}
\end{table}

\jissec{On /{\ipaET}j/ or /\ipaET/}
One of the doubtful phonemes in Deẽreẽ’s phonology is the grapheme <ẽi>’s pronunciation. It can be
pronounced with palatalization or not, that is, /{\ipaET}j/ or /\ipaET/. How to pronounce it is
normally determined by its position in the word.

As the last phoneme of a word <ẽi> will be palatalized: /{\ipaET}j/. But when <ẽi> is initial or
inside a word, it will be pronounced as either palatalized or not. When followed by j, l, s, n, p,
t, or sh, that is a consonant from bilabial to palatal but not further back (k, g, r), it is
palatalized, otherwise it is not.

\jisec{Stress}
Stress in words is basically on the last syllable.
\begin{itemize}
\item zhalbar\emph{ki} (beer)
\item loj\emph{shë} (history)
\end{itemize}

\newgle{ojmata}{to predict}
\newgle{tëjete}{to forget}
However, stress on infinitive verbs is on the second-to-last syllable, for example with
\gls{tëjete}, \gls{drete} and \gls{ojmata}.
\begin{itemize}
\item të\emph{je}te (to forget)
\item \emph{dre}te (to go)
\item oj\emph{ma}ta (to predict)
\end{itemize}

\newgle{forfof}{desert}
\newgle{aù}{little/small}
And a when a word takes a person suffix, this does not change stress location. Examples with
\gls{forfof} and \gls{aù}.
\begin{itemize}
\item for\emph{fof}’a (the desert)
\item a\emph{ùr}’mer (these little ones)
\end{itemize}

\jisec{Phonotactics}
The syllable structure is \textbf{(C|B)V(L)(F)} where:
\begin{description}
\item[C] is any consonant
\item[B] is a group of two consonants among \emph{tr, dr, pl, bl, kl, gl, kr, gr}
\item[V] is any vowel
\item[L] is a consonant among \emph{j, l, r, s}
\item[F] word-final only, a consonant among \emph{n, p, t, k, f, s, sh, l}
\end{description}

\jissec{Vowel pairs}
Quite often, two vowels will appear in a word consecutively. In this case, there are three cases:
long vowel, diaeresis, or semi-vowel.

\jisssec{Long vowel sounds}
In a word like \newgle{aal}{wind}\gls{aal}, There is a long vowel sound due to vowel reduplication.
There, it is pronounced as [a.al], instead of [a:l]. That is, the real pronounciation of this word
is as close to diaeresis as possible, as opposed to a ‘real’ long vowel.

\jisssec{Diaeresis}
\newgle{tëfuete}{to wipe}
\newgle{fùã}{powder}
Two consecutive vowels will give a diaeresis, that is, the two vowels being pronounced separately
without any semi-vowel, in words like \gls{fùã}, powder, or before a verb ending, for example in
\gls{tëfuete}, to wipe.

\jisssec{Semi-vowel}
\newgle{marùëlf}{nature}
A semi-vowel, that is, a sound like /j/, /w/, or /ɥ/ (the \emph{u} in the French huit), is used in
various words, for example the Deẽreẽ word for nature, \gls{marùëlf}.



%-----
\jichap{Morphology}\label{chap:morphology}
This chapter, morphology, aims to describe \emph{what the words look like}.
This means describing the nouns, adjectives, adverbs, verbs in their basic forms.

\jisec{Gender}
Deẽreẽ has three genders: Human, marked \emph{H}, Magic, marked \emph{M}, and Common, marked
\emph{C}. Every word has a gender, whatever its grammatical nature (part of speech) might be: nouns
have gender, but verbs have gender too, as well as adjectives. Words that are not nouns do not just
agree in gender (if they do), but has their own intrinsic categories.

To determine the gender of a word, you have to look at its last vowel (whether or not it ends in a
vowel). The rules are in table \ref{tab:morph-genders}.

\begin{table}[h]\label{tab:morph-genders}
\begin{center}
\begin{tabular}{|c|c|c|c|}
\hline
\textbf{Human}  & e & ẽi & u\\\hline
\textbf{Magic}  & a & ã  & o\\\hline
\textbf{Common} & ù & ë  & i\\\hline
\end{tabular}
\end{center}
\caption{Vowels in the Deẽreẽ genders (H:e/ẽi/u, M:a/ã/o, C:ù/ë/i)}
\end{table}

For example:
\begin{description}
\newgle{zhalbarki}{beer}\newgle{drete}{to go}\newgle{magra}{soul}
\item[\gls{zhalbarki}] beer, ending in \emph{i}, is of \emph{Common} gender
\item[\gls{drete}] to go, ending in \emph{e}, is of \emph{Human} gender
\item[\gls{magra}] soul, ending in \emph{a}, is of \emph{Magic} gender
\end{description}


\jisec{Number}
Number in Deẽreẽ is simply singular (\emph{SI}) and plural (\emph{PL}). As gender is marked with the
last vowel of a given word, number is marked with the last consonant.

\jissec{Formation}
\begin{table}[h]\label{tab:morph-number}
\begin{center}
\begin{tabular}{|c|ccccccccc|}
\hline
\textbf{Singular} & -l & -n & -p & -t & -k & -f & -s & -sh & -\emph{vowel}\\\hline
\textbf{Plural}   & -r & -m & -b & -d & -g & -v & -z & -zh & -r\\\hline
\end{tabular}
\end{center}
\caption{Number (singular and plural) formation}
\end{table}

For example, magra, magrar; aal; aar; \newgle{sùn}{metal}\gls{sùn} (metal), sùm, \newgle{zërsh}{pig}
\gls{zërsh} (pig), zërzh.

\jissec{Meaning and agreement}
\jisssec{Nouns}
The meanings of singular and plural for nouns are as expected: singular means that the thing is
present once, plural that it is present several times: \newgle{zheo}{fairy}\gls{zheo}, fairy,
\gls{zheo}r, fairies.

As for uncountable nouns, singular means ‘some stuff’, while plural means ‘several kinds of the
stuff’. This also is quite an expected behavior: \newgle{lojshë}{history}\gls{lojshë}, history,
some part of history; \gls{lojshë}r, histories, or \newgle{geni}{milk}\gls{geni}, milk, some milk;
\gls{geni}r, several kinds of milk.

\jissec{Verbs}
Verbs agree in number with the agent. See chapter Conjugation on page \pageref{chap:conjugation} for
more detail.

\jissec{Adverbs and adjectives}
These do not agree, as is explained in chapter Syntax (page \pageref{chap:syntax}).


\jisec{Numerals}
This section discusses how to count in Deẽreẽ. The language uses a \emph{base 12} counting system.

\jissec{Cardinal numbers}
The terms for numbers 1 to 11 are given in the table below:

\begin{center}\begin{tabular}{|l|c|c|c|c|c|c|c|c|c|c|c|}\hline

\textbf{Number} & 1 & 2 & 3 & 4 & 5 & 6 & 7 & 8 & 9 & 10 & 11 \\\hline
\textbf{Deẽreẽ} & ki & miki & as & toni & taki & mias & mibe & tol & kani & üma & moi\\\hline

\end{tabular}\end{center}

Powers of 12 and 0 have other meanings in the language:

\begin{center}\begin{tabular}{|l|c|c|c|c|c|}\hline

\textbf{Number} & $0$ & $12$ & $12^2 = 144$ & $12^3 = 1728$ & $12^4 = 20736$ \\\hline
\textbf{Deẽreẽ} & dal & shëi & lül & zeẽ & kënak \\\hline
\textbf{Meaning} & nothing & house & village & city & palace \\\hline

\end{tabular}\end{center}

Numbers are constructed with the conjunction \emph{ak}, ‘and’, and the word \emph{kon}, ‘number, quantity’. For example:

\begin{description}
\item[$42 = 3 \times 12 + 6$] as shëi ak mias kon
\item[$200 = 144 + 4 \times 12 + 8$] lül ak toni shëi ak tol kon
\end{description}

\jissec{Ordinal numbers}
\todo{ordinals}

\jissec{Partital numbers}
\todo{fractions}




%-----
\jichap{Derivational Morphology}\label{chap:derivational-morphology}
The name ‘derivational morphology’ refers to all the mechanisms of word derivation, that is, how to
make a word out of another. These rules apply to several kinds of words, and are hereafter grouped
by part of speech they derive.

%---
\jisec{Derivations from nouns}
\jissec{Owner/Ruler: \emph{-(s)ket}}
\jisssec{Examples}
kortis (shop), kortisket (shopkeeper); lül (village), lülket (village mayor); beet (sheep), beesket (sheperd)

\jisssec{Etymology}
Comes from the verb \emph{oskete}, to own. This suffix is a reduction of the phrase \emph{<noun> osket sep}.

%---
\jisec{Derivations from adjectives}
\jissec{To turn into, to become X: \emph{-fütü}}
\jisssec{Examples}
dreer (cold), dreerfütü (to get cold); lëlü (big), lëlfütü (to grow); zaf (clean), zafütü (to clean)

\jisssec{Etymology}
From the verb \emph{fütü}, to push, or to turn into. The phrase normally used for nouns is \emph{<noun> ol fütü}.

%---
\jisec{Derivations from verbs}
\jissec{Tool: \emph{-<V>sën}}
\jisssec{Examples}
shëskata (to charm), shëskasën (charisma); ãgrete (to rule), ãgresën (authority, personality trait); kütü (to see), küsën (attention, prudence)

\jisssec{Etymology}
\todo{etymology of -sën}

%---
\jissec{Result: \emph{-ür}}
\jisssec{Examples}
dekete (to write), dekür (writings); tomete (to look like), tomür (appearance); ramata (to hurt), ramür (pain)

\jisssec{Etymology}
The postposition \emph{dür} (after), and the phrase \emph{<verb-infinitive> dür}.

%---
\jissec{Performer: \emph{-ẽi}}
\jisssec{Examples}
fete (to drink), fetẽi (drinker); bapete (to tell a lie), bapẽi (liar)

\jisssec{Etymology}
The word \emph{ẽi} (person).


%-----
\jichap{Morphosyntax}\label{chap:morphosyntax}
This chapter discusses inflexions of words in Deẽreẽ.

\jisec{Person suffixes}
Person suffixes in Deẽreẽ are suffixes that can be added to different kinds of words, notably nouns,
verbs, adjectives, adverbs, or postpositions. Their purpose is to precise which person the word
refers to. The language uses six grammatical persons, the usual ones:

\begin{description}
\item[1.SI] I, speaker
\item[2.SI] singular you (thou), whom is spoken to
\item[3.SI] he/she/it, neither speaker nor person being spoken to
\item[1.PL] we, several people including the speaker
\item[2.PL] plural you, several people including the one(s) spoken to
\item[3.PL] they, several people neither speaking nor being spoken to
\end{description}

\jissec{Form and agreement}
Person suffixes have to agree with what they \emph{refer to}. This is very important because, the
word they refer to is not always the word on which they are. They decline in \emph{number},
\emph{person}, and \emph{gender}. Their basic forms, preceded by an apostrophe, are listed in table
\ref{tab:morph-basic-pers-suff}.

\begin{table}[h]
\begin{center}\label{tab:morph-basic-pers-suff}
\begin{tabular}{|c||c|c|c|c|}
\hline
\multirow{2}{*}{\textbf{Gender}} & \multicolumn{2}{|c|}{\textbf{Human}} & \multirow{2}{*}{\textbf{Magic}} & \multirow{2}{*}{\textbf{Common}}\\
                                 & \textbf{Female/neuter} & \textbf{Male} & & \\\hline\hline
\textbf{1.SI}                    & ’esh & ’ẽish & ’ash & ’ùsh \\\hline
\textbf{2.SI}                    & ’el  & ’ẽil  & ’al  & ’ùl  \\\hline
\textbf{3.SI}                    & ’e   & ’ẽi   & ’a   & ’ù   \\\hline
\textbf{1.PL}                    & ’ezh & ’ẽizh & ’azh & ’ùzh \\\hline
\textbf{2.PL}                    & ’er  & ’ẽir  & ’ar  & ’ùr  \\\hline
\textbf{3.PL}                    & ’er  & ’ẽir  & ’ar  & ’ùr  \\\hline
\end{tabular}
\end{center}
\caption{Person suffixes in their basic form}
\end{table}

We have to notice there is no difference between \textbf{2.PL} and \textbf{3.PL}. The difference is
thus done with context understanding.

\jissec{Definite}
The definite article, that is, the equivalent of \emph{the}, is formed simply by applying the person
suffix after the noun, in their basic form from table \ref{tab:morph-basic-pers-suff}. For example
see (\ref{exe:person-def}), with \newgle{mosoj}{cat}\gls{mosoj}, cat.

\begin{exe}
\ex\label{exe:person-def}
\begin{xlist}
\ex\gll mosoj’a\\
cat(MAG)-3.SI.MAG\\
\trans ‘the cat’

\ex\gll mosoj’ash\\
cat(MAG)-1.SI.MAG\\
\trans ‘I, the cat’

\ex\gll mosojr’ar\\
cat(MAG)-2.PL.MAG\\
\trans ‘you, cats’
\end{xlist}
\end{exe}

As we see in this example (\ref{exe:person-def}), the person suffixes add meaning to the words in a
short, efficient way. They are also used with verbs, as in \emph{dret’ẽil}, I go. Details on
conjugation are however in a separated chapter.

\jissec{Indefinite}
Indefinite articles express the idea that the speaker doesn’t know \emph{which} thing they are
speaking about, but \emph{a} thing. Its form in Deẽreẽ is a separate word, not a suffix. This word
is \newgle{et}{indefinite article}\emph{et}, which declines as described in table \ref{tab:morph-indef-pers-suff}.

\begin{table}[h]
\begin{center}\label{tab:morph-indef-pers-suff}
\begin{tabular}{|c||c|c|c|c|}
\hline
\multirow{2}{*}{\textbf{Gender}} & \multicolumn{2}{|c|}{\textbf{Human}} & \multirow{2}{*}{\textbf{Magic}} & \multirow{2}{*}{\textbf{Common}}\\
                                 & \textbf{Female/neuter} & \textbf{Male} & & \\\hline\hline
\textbf{1.SI}                    & etesh & etẽish & et’ash & et’ùsh \\\hline
\textbf{2.SI}                    & etel  & etẽil  & et’al  & et’ùl  \\\hline
\textbf{3.SI}                    & et    & ẽit    & eat    & eùt    \\\hline
\textbf{1.PL}                    & etezh & etẽizh & et’azh & et’ùzh \\\hline
\textbf{2.PL}                    & eter  & etẽir  & et’ar  & et’ùr  \\\hline
\textbf{3.PL}                    & ed    & ẽid    & ead    & eùd    \\\hline
\end{tabular}
\end{center}
\caption{Indefinite article \emph{et} and its declensions}
\end{table}

The position of the indefinite article is still after the noun it describes. See example
\ref{exe:person-indef}.

\begin{exe}
\ex\label{exe:person-indef}
\begin{xlist}
\ex\gll \gls{eltol} eat\\
bird(MAG) 3.SI.MAG.INDEF\\
\trans ‘a bird’

\ex\gll \newgle{shil}{worm}\gls{shil} et’ùl\\
worm(COM) 2.SI.COM.INDEF\\
\trans ‘You worm!’
\end{xlist}
\end{exe}

\jissec{Demonstrative}
Without a pretty table this time; demonstrative articles are used to designate a specific object out
of a group, as a transition from indefinite to definite. The English equivalent are \emph{this} and
\emph{that}.

These are of two types: proximal and distal.

\jisssec{Proximal demonstrative}
The word proximal means \emph{close}, so a proximal demonstrative is an article designating a thing
close to the speaker (\emph{this}). In Deẽreẽ, it is formed with the definite suffixes preceded with
the letter \emph{m}.

\begin{exe}
\ex\label{exe:person-prox-dem}
\gll mosoj’ma\\
cat(MAG)-3.SI.PDEM\\
\trans this cat
\end{exe}

\jisssec{Distal demonstrative}
As proximal means close, distal means \emph{remote}. It is so the equivalent of \emph{that}. It is
made in the language with the definite suffixes preceded with \emph{asm}. It is the occasion here to
introduce the word \newgle{senù}{fish}\gls{senù}, fish.

\begin{exe}
\ex\label{exe:person-dist-dem}
\gll \gls{senù}’asmù\\
fish(COM)-3.SI.PDEM\\
\trans that fish
\end{exe}

\jissec{Genitive}
’re
\jisssec{Owner}
todel’rẽi
\jisssec{Distinct}
ëj adal’e ãblẽi’re


\jisec{Adjectives}

They come before the noun, and do not agree with it in any way.

\jissec{Comparatives and Superlatives}
Comparatives are ways of expressing that something is *more*, *less* or *as much X as* something else.

The comparative in Deẽreẽ is placed before the comparee, as a qualifier.
It is constructed in three parts: \textbf{reference-scale-direction}.

\begin{description}
\item[The reference] followed by \emph{nal}, ‘through’, is what the comparee will be compared to.
\item[The scale] states on what criteria the comparison will be.
\item[The direction] indicates if the comparee is more, less or equal to the reference in the given scale.
\end{description}

\begin{table}[h]
\begin{center}
\begin{tabular}{|llll|l|}\hline

\textbf{Reference} & \textbf{Scale} & \textbf{Direction} & \textbf{Comparee} & \textbf{Translation} \\\hline

shükẽ’me \emph{nal} & lëlü & & azhe’me & this woman is \textbf{taller} than this man \\\hline
samis eat \emph{nal} & dreri & \textbf{rar} & edan eat & a river is \textbf{less} wide than a lake \\\hline
ãda’a \emph{nal} & lif & \textbf{sho} & zhok’a & the blood is \textbf{equally as} red as the rose \\\hline

\end{tabular}
\end{center}
\caption{Comparatives}
\label{tab:morphology-comparatives}
\end{table}

Table \ref{tab:morphology-comparatives} shows through examples how the comparative is constructed. Vocabulary used in this table:

\begin{description}
\item[shükẽ] man (male)
\item[azhe] woman
\item[lëlü] big, tall
\item[samis] lake
\item[edan] river
\item[dreri] wide
\item[ãda] rose (flower)
\item[zhok] blood
\item[lif] red
\end{description}

Table \ref{tab:morphology-comparatives} shows well the role of the postposition \emph{nal}:
its meaning is, ‘relative to’, ‘by reference to’.
Hence a sentence like \emph{shükẽ’me nal lëlü azhe’me} can be litterally translated as:
‘Relative to this man, this woman is tall.’, meaning ‘This woman is taller than this man.’ (See vocabulary list above).

A superlative is constructed the same way as a comparative, except that we compare with everything: to be the \emph{best} means to be \emph{better than everything}.

\begin{exe}
\ex\label{ex:morphology-superlatives}

\textbf{Un’ur nal mama aso ẽi’esh sep are’sh!}\\
/'un.u{\ipaR} nal 'ma.ma 'a.so '{\ipaET}i.e{\ipaS} s{\ipaE}p a.'{\ipaR\ipaE\ipaS}/

\gll un-’ur nal mama aso esh sep are-’sh\\
all-3PL CMP very good 1S SUBC want-1S\\
\trans ‘I wanna be the very best!’

\end{exe}

The superlative of example \ref{ex:morphology-superlatives} litterally means, ‘Relative to everything, me be very good, I want’.


\jisec{Verbs}
Here I’ll explain the different tense/aspect grams in \emph{Deẽreẽ}. We’ll use three example verbs in this chapter: \emph{dekete}, ‘to write’; \emph{trasata}, ‘to give’, and \emph{aütü}, ‘to come’.

\jissec{Negation}
Negation of a verb is done by putting the word \emph{rar}–also meaning ‘no’–after the verb.

\jissec{Simple tenses}
To construct the present simple, one has to apply a pronoun suffix to the verb root (which ends in \emph{et/at/üt}).
However if the verb’s gender is the same as the pronoun’s gender and this suffix is indefinite,
the verb’s final \emph{t} and the suffix’s vowel may be ommited.
See table \ref{tab:conj-present-simple}.

\begin{table}[h]
\begin{center}
\begin{tabular}{|c|c|c|c|}\hline

\textbf{Verb}   & dekete (H) & trasata (M) & aütü (C) \\\hline
\textbf{Human}  & deke’sh, deke’l, deke & trasat’esh, trasat’el, trasat’e & aüt’esh, aüt’el, aüt’e \\\hline
\textbf{Magic}  & deket’ash, deket’al, deket’a & trasa’sh, trasa’l, trasa & aüt’ash, aüt’al, aüt’a \\\hline
\textbf{Common} & deket’üsh, deket’ül, deket’ü & trasat’üsh, trasat’ül, trasat’ü & aü’sh, aü’l, aü \\\hline

\end{tabular}
\end{center}
\caption{Examples of \emph{dekete}, \emph{trasata} and \emph{aütü} in the present simple tense, 1S, 2S \& 3S}
\label{tab:conj-present-simple}
\end{table}

\begin{description}
\item[deke’sh] I write.
\item[trasat’el rar] You do not give.
\item[aü rar] It does not come.
\end{description}

The past simple is constructed in a manner similar to the present simple, except that here the verb endings \emph{et/at/üt}
are replaced with \emph{ẽi/oi/ëi}.
See table \ref{tab:conj-past-simple}.

\begin{table}[h]
\begin{center}
\begin{tabular}{|c|c|c|c|}\hline

\textbf{Verb}   & dekete (H) & trasata (M) & aütü (C) \\\hline
\textbf{Human}  & dekẽi’sh, dekẽi’l, dekẽi & trasoi’esh, trasoi’el, trasoi’e & aëi’esh, aëi’el, aëi’e \\\hline
\textbf{Magic}  & dekẽi’ash, dekẽi’al, dekẽi’a & traoi’sh, trasoi’l, trasoi & aëi’ash, aëi’al, aëi’a \\\hline
\textbf{Common} & dekẽi’üsh, dekẽi’ül, dekẽi’ü & trasoi’üsh, trasoi’ül, trasoi’ü & aëi’sh, aëi’l, aëi \\\hline

\end{tabular}
\end{center}
\caption{Examples of \emph{dekete}, \emph{trasata} and \emph{aütü} in the past simple tense, 1S, 2S \& 3S}
\label{tab:conj-past-simple}
\end{table}

\begin{description}
\item[dekẽi’sh rar] I did not write
\item[trasoi] It(M) gave
\item[aëi’e] He/She came
\end{description}

\jissec{Complex tenses}
The previous tenses were simple tenses. The tenses in this section will require a \emph{verb participle} to be constructed.

The verb participle is constructed by replacing the infinitive ending with either \emph{eu, aã} or \emph{üi}, depending on gender. Note that those are \emph{not} diphthongs, and should not be pronounced as such.\\

Continuing our example:
\begin{description}
\item[dekete] \emph{dekeu} /de.'ke.u/
\item[trasata] \emph{trasaã} /t{\ipaR}a.'sa.ã/
\item[aütü] \emph{aüi} /a.'y.i/
\end{description}

The complex tenses are constructed using the verb’s participle followed by a conjugated auxilary verb, either \emph{sete}, ‘to do’, or \emph{drete}, ‘to go’.
See table \ref{tab:conj-complex-tenses} for reference.

\begin{table}[h]
\begin{center}
\begin{tabular}{|c|c|c|}\hline

\textbf{Tense of auxilary} & \textbf{Present} & \textbf{Past} \\\hline
PTCP + sete & present progressive & past progressive \\\hline
PTCP + drete & future & past inchoative \\\hline

\end{tabular}
\end{center}
\caption{Complex tenses depending on auxilary verb and tense}
\label{tab:conj-complex-tenses}
\end{table}

When a verb in a complex tense is negated, the word \emph{rar} must come \textbf{between} the participle and the auxilary.

Examples of the various tenses:
\begin{description}
\item[aüi se’a] it(M) is coming (present progressive)
\item[dekeu sẽi’sh] I(H) was writing (past progressive)
\item[aüi rar dre’sh] I(H) will not come (future)
\item[trasaã rar drẽi’l] you(H) did not start to give (past inchoative)
\end{description}

\jissec{The imperative/prohibitive}
Orders and interdictions in Deẽreẽ are constructed with a prefix \textbf{a(r)-} added to the verb.
The person suffix stays the same, but there is only two tenses allowed: the present simple and the future.
Future imperatives are \emph{weaker} than present simple imperatives. They are both less compelling and more polite.

\begin{description}
\item[adeke’l!] write!
\item[adekeu rar dre’l] Do not write in the future.
\item[atrasat’ezh] let us give!
\item[araüt’el rar] come not!
\end{description}

\jissec{Modality}
Verbal modality changes the implied meaning of a verb form in two possible ways:
is this action \emph{possible} (epistemic modality); is this action \emph{allowed} (deontic modality).
It is expressed in English with the auxilaries ‘could’, ‘should’, ‘may’, etc.

See table \ref{tab:verbal-modality}.

\begin{table}[h]
\begin{tabular}{|c|c|p{3cm}|p{3cm}|p{3cm}|}\hline
\textbf{Expressed aspect} & \textbf{Construction} & \textbf{Example} & \textbf{Translation} & \textbf{Comment} \\\hline\hline
Certainty & verb + oi & \emph{‘Ü apiske’l oi!’} & ‘Throw it away!’ & Often used with the imperative \\\hline
“Maybe” & verb + garai & \emph{‘Bape’r garai’} & ‘They may be lying.’ & — \\\hline
Impossibility & verb + rar & \emph{‘Aüt’esh rar’} & ‘I do not come.’ & Is also negation \\\hline
\hline
Ability & verb + sep onütü & \emph{‘Se’zh sep onüt’ezh!’} & ‘We can do it.’ & Means also ‘authorised’ \\\hline
Unability & verb + sep onütü rar & \emph{‘Krüt borüg’ür süp onü’r rar.’} & ‘The animals cannot eat.’ & — \\\hline
\hline
Obligatory & ãgreu + verb & \emph{‘Ãgreu paküt’esh.’} & ‘I have to dig.’ & — \\\hline
Forbidden & ãgreu + verb + rar & \emph{‘Ãgreu me’r rar.’} & ‘We are not allowed to laugh.’ & — \\\hline
\hline
Wanted & verb + sep arete & \emph{‘Shügrat’esh el sap are’sh.’} & ‘I want to get close to you’ & — \\\hline
Unwanted & verb + sep arete rar & \emph{‘Rüdüt’el esh süp are’sh rar.’} & ‘I don’t want you to hurt me.’ & — \\\hline
Indifferent & verb + sep debete & \emph{‘Trat’e rar sap debe’sh.’} & ‘I don’t mind that he cannot read.’ & — \\\hline
Needed & verb + sep üis-\textsc{GEN} & \emph{‘Mëiüt’esh süp üis’resh.’} & ‘I need to cry.’ & — \\\hline
\hline
Appreciated & sãkreu + verb & \emph{‘Sãkreu braset shükẽ’e.’} & ‘The man likes to walk.’ & — \\\hline
Unappreciated & sãkreu + verb + rar & \emph{‘E sãkreu kü rar.’} & ‘He doesn’t like to see her.’ & — \\\hline
\end{tabular}
\caption{Verbal modality in Deẽreẽ}
\label{tab:verbal-modality}
\end{table}





%-----
\jichap{Syntax}\label{chap:syntax}
\jisec{Noun phrase}
\emph{Deẽreẽ} is a \textbf{head-final} language. This means that in most cases if not all, the modifier words or clauses will come \emph{before} the modified item. This is true of noun-adjective order.

\begin{exe}
\label{ex:syntax-noun}
\ex
\gll lif zhok-’a\\
red blood-3S\\
\trans ‘The red blood’
\ex
\gll aso kal eat\\
good eye 3S.INDF\\
\trans ‘a good eye’
\end{exe}


\jisec{Sentence}
This section will discuss the ordering of parts of a sentence, and the morphemes needed for certain specific constructions, such as subclauses. Let us define a few notations first: \textbf{V} for Verb, \textbf{O} for the Objects of the verb, both direct and indirect, and \textbf{S} for Subject of the verb.

\jissec{Primary order}
The basic sentence word order is \textbf{OVS}.

\begin{exe}
\ex\label{ex:syntax-primary-flower}
\textbf{Ãdiir ead trasat shükẽ’e}\\
/{\ipaAT}.'dii{\ipaR} e.'ad t{\ipaR}a.'sat 'shy.k{\ipaET}.e/

\gll ãdiir ead trasat shükẽ-’e\\
flower 3S.INDF give man-3S\\
\trans ‘The man gives flowers.’

\ex\label{ex:syntax-primary-ghost}
\textbf{Mog’ar parat naiës’rel}\\
/'m{\ipaO}g.ar pa.'{\ipaR}at nai.'əs.{\ipaR\ipaE}l/

\gll mog-’ar parat naiës-’rel\\
ghost.PL-3PL toFear child-2SG.GEN\\
\trans Your child fears the ghosts.
\end{exe}

\jissec{Secondary order}
The secondary order of words, is secondary by rank. It is fairly often used, but only in particular constructs that will be treated below.

This word order is verb-first. So it can be either \textbf{VOS}, if the subject \textbf{S} is not attached to the verb in the form of a pronoun suffix. It gives \emph{V O S}.
If the subject \emph{is} a pronoun suffix on the verb, then the word order can be noted \textbf{VSO} : \emph{V’S O}.
In this latter case, the subject being attached to the verb, we can say that there is no fully described subject, and write this word order as \textbf{V(’S)O(S)}.
The chosen notation will be \textbf{VOS}.

Clauses in this order can still have adverbial modifiers come before the verb.


\jisec{Subclauses: \emph{sep/sap/süp}}
Subclauses use the secondary word order, \textbf{VOS}, and an additionnal morpheme, the particle \emph{sep/sap/süp}, depending on gender (resp. Human, Magic, Common). In the following examples, this particle will be glossed as SUBC.

Its construction will be described below, depeding on whether the subclause is nominalized or not, that is whether it acts as a noun or qualifies a noun.

\jissec{Subclause as a noun}
Subclauses may be used as nominal constructions, in which case they may include a subject, and an object if required by the verb.

We see in example \ref{ex:syntax-subclause-as-noun-1} that there is gender agreement, between the subclause particle \emph{sep} and
the verb in the clause, \emph{assete}, meaning, ‘to light’.
\begin{exe}
\ex\label{ex:syntax-subclause-as-noun-1}
\textbf{Asset milzhër’ü Nosh’a sep küt’esh.}\\
/as.'s{\ipaE}t mil.'{\ipaZ}ər.y 'no\ipaS.a s{\ipaE}p 'kyt.e\ipaS/

\gll asset milzhër-’ü Nosh-’a sep küt-’esh\\
toLight(H) forest-3S Moon-3S SUBC(H) see-1S\\
\trans ‘I see that the Moon lights the forest.’
\end{exe}

In example \ref{ex:syntax-subclause-as-noun-2}, the clause is used as a noun and followed by the benefactive postposition \emph{ol}.
Once again, there is agreement between \emph{sep} and the verb \emph{dekete}, ‘to write’.
\begin{exe}
\ex\label{ex:syntax-subclause-as-noun-2}
\textbf{Deke sep ol trasat’esh.}\\
/d{\ipaE}.'k{\ipaE} s{\ipaE}p '{\ipaO}l t{\ipaR}a.'sat.{\ipaE\ipaS}/

\gll deke sep ol trasat-’esh\\
write(H).3S SUBC(H) to give-1S\\
\trans ‘I give to the person who writes.’
\end{exe}

\jissec{Subclause as a qualifier}
Whenever a subclause qualifies a noun, the verb inside the subclause takes the qualified noun either as its subject or its object, direct or indirect.

In all three cases (subject, direct object, indirect object), what is missing in the subclause will be identified by a pronoun in the \emph{vocative} case (\emph{le/la/lü}).

This vocative pronoun will be inside the subclause (between the verb and \emph{sep/sap/süp}) in the first two cases, and directly after, on a postposition, in the case of an indirect object.\\

When the subclause is \textbf{missing subject}, a subject is placed on the verb in the subclause, as a vocative pronoun suffix.
Both this pronoun and the subclause particle agree in gender with what the subject actually is, in the case of example (\ref{ex:syntax-subclause-missing-S}),
\emph{Nosh’a}, ‘the Moon’.

\begin{exe}
\ex\label{ex:syntax-subclause-missing-S}
\textbf{Asset’la sap Nosh’a küt’esh.}\\
/as.'s{\ipaE}t.la sap 'n{\ipaO\ipaS}.a 'kyt.{\ipaE\ipaS}/

\gll asset-’la ezh sap Nosh-’a küt-’esh\\
toLight-3S(M).VOC 1PL SUBC(M) Moon(M)-3S(M) see-1S\\
\trans ‘I see the moon that lights us.’
\end{exe}

The \textbf{direct object} is missing in example \ref{ex:syntax-subclause-missing-DO}. A vocative placeholder is used within the subclause where an object
would be.
The vocative pronoun \emph{lü} as well as the subclause particle \emph{süp} agree in gender with the actual object, located after the subclause, \emph{ãdiil’ü}, ‘the flower’.

\begin{exe}
\ex\label{ex:syntax-subclause-missing-DO}
\textbf{Trasat’e lü süp ãdiil’ü küt’esh.}\\
/t{\ipaR}a.'sat.e ly syp {\ipaAT}.'diil.y 'kyt.{\ipaE\ipaS}/

\gll trasat-’e lü süp ãdiil-’ü küt-’esh\\
give-3S 3S(C).VOC SUBC(C) flower(C)-3S(C) see-1S\\
\trans ‘I see the flower s/he gives.’
\end{exe}

In the case of an indirect object missing from the subclause (example \ref{ex:syntax-subclause-missing-IO}), the agreement is still the same:
both \emph{sap} and the vocative \emph{’la} agree with the actual indirect object, \emph{sãs’a}.
However, the vocative is not within the subclause, but on the postposition \emph{after} the subclause particle.

\begin{exe}
\ex\label{ex:syntax-subclause-missing-IO}
\textbf{Aüt eltol’a sap ëi’la sãs’a küt’esh.}\\
/a.'yt {\ipaE}l.'tol.a sap 'əi.la 's{\ipaAT}s.a 'kyt.{\ipaE\ipaS}/

\gll aüt eltol-’a sap ëi-’la sãs-’a küt-’esh\\
come bird-3S SUBC(M) from-3S(M).VOC tree(M)-3S(M) see-1S\\
\trans ‘I see the tree from which the bird comes.’
\end{exe}


\jisec{Questions}
The syntax of questions requires a question particle, depending on gender:
\textbf{lu, lo, li} for \emph{Human}, \emph{Magic} and \emph{Common} respectively.
They are not to be confused with the vocative suffixes \emph{’le, ’la, ’lü}.

Questions use the \textbf{secondary word order}, (VOS), and the question particle can be inside or at the end of the sentence.

\jissec{Yes/No questions}

\begin{exe}
\ex\label{ex:syntax-yes-no-question}
\textbf{Küt’el tor’a ron eltol’a li?}\\
/'kyt.{\ipaE}l 't{\ipaO\ipaR}r.a {\ipaR\ipaO}n eltol’a li/

\gll küt-’el tor-’a ron eltol-’a li\\
see-2S mountain-3S on bird-3S Q\\
\trans ‘Do you see the bird on the mountain?’
\end{exe}

In example \ref{ex:syntax-yes-no-question}, we see that the verb is in first position,
followed by an object (\emph{tor’a ron eltol’a}, ‘the bird on the mountain’),
and the question particle \emph{li} comes in the end.\\

In a yes/no (or closed) question, the question particle is located at the end of the sentence and must agree in gender with the verb.

\jissec{Open questions}
An open question is one that would use a wh- word in English (what, who, when, etc.).
In Deẽreẽ, those are constructed with the VOS word order, and the question particle is used as (and called) a placeholder for what is asked for.
This placeholder can take qualifiers in the form of adjectives or nouns, that serve to discriminate if we ask about nature of thing, time, location, etc.
The placeholder will also take a person suffix, making it well distinct from the closed question particle.

Compare:

\begin{description}
\item[Feu dre’l le?] Are you drinking?
\item[Feu dre’l li’ü] What are you drinking?
\end{description}

\emph{le} in the first sentence is the question particle, whereas in the second sentence \emph{li} is an interrogative placeholder.
The placeholder is in the \emph{Common} gender to agree with an expected response, whose gender is yet unknown.

\begin{exe}
\ex\label{ex:syntax-open-question-1}
\textbf{Fet zhalbraki’ü lu’mer?}\\
/f{\ipaE}t {\ipaZ}al.'b{\ipaR}a.ki.y 'lu.m{\ipaE\ipaR}/

\gll fet zhalbraki-’ü lu-’mer ?\\
drink beer-3S Q(H)-2PL.DEM ?\\
\trans ‘Who of you drinks a beer?’
\end{exe}

Example \ref{ex:syntax-open-question-1} shows how the interrogative placeholder agrees with the expected answer:
its suffix is a second person demonstrative, indicating the question is asked to a group of people.





%-----
\appendix
\jichap{Samples of the language}
\jisec{Nalkor un’rar Dësish’ü — The King of All Snakes}
\label{samp:theKingOfAllSnakes}
\emph{This was the first ‘poem’ written in Deẽreẽ.}

Edan’ra mete’e asabre’l,\\
Nusha’r noshi i lavësã sap aküt’el,\\
Lëlü aal’ra eltol’ra nën’ü atoüt’el,\\
Adal tis’mü’rü mig’ür ak ãdiir’ar asãgre’l.

Kerpo pitëk’ü i üsaat nalkor un’rar dësish’ü.\\
Sabre’l sep un’e, küt’el süp un’ü, toüt’el süp un’ü, sãgre’l sep un’e, osket dësish’ü.\\
Ãgre dësistis’rü sep trẽshet rar kidren’mü.\\
Pa vilẽ’rü nãilat greba’r sap un eltor’ar.

\jisssec{English translation}
Listen to the river’s laugh,\\
See in the sky the clouds swim,\\
Know the calm of the great wind bird,\\
Love the riches and flowers of this old place.

In the yellow mangrove lives the king of all snakes.\\
Everything you hear, everything you see, everything you know, everything you love, belongs to the king.\\
This noble being hasn’t finished ruling his kingdom.\\
Every bird that flies sings his just name.



%-----
\backmatter
\clearpage\glsaddall
\begin{multicols}{2}\printglossaries\end{multicols}
\listoffigures\addcontentsline{toc}{chapter}{\listfigurename}
\listoftables\addcontentsline{toc}{chapter}{\listtablename}
\listoftodos
\end{document}

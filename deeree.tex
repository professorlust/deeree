\documentclass[a4paper, oneside]{book}
\usepackage{jipkg}
\usepackage{multirow}

%few ipa symbols
\newcommand{\ipaE}{ε}
\newcommand{\ipaET}{ɛ̃}
\newcommand{\ipaO}{ɔ}
\newcommand{\ipaAT}{ɑ̃}
\newcommand{\ipaS}{ʃ}
\newcommand{\ipaZ}{ʒ}
\newcommand{\ipaR}{ʁ}

%glossary
\usepackage[toc]{glossaries}
\newcommand{\newgle}[2]{\newglossaryentry{#1}{name=#1, description={#2}}}
\glossarystyle{listgroup}
\makeglossaries

%gloss and examples
\usepackage{gb4e}

\title{Deẽreẽ\\Grammar of a fantasy constructed language}
\author{Jirsad \textsc{Ïoh}}
\date{2016 — \today}
\jihypersetup{Deẽreẽ — A fantasy conlang}{Jirsad ÏOH}

\begin{document}
\selectlanguage{english}
\frontmatter
\maketitle
\tableofcontents
\jichap{Introduction}
\begin{exe}
\ex\label{ex:intro-cat}
\textbf{Fãssa mosoi’a.}\\
/'fã.s:a mo.'soi.a/\\
\gll fãssa mosoi-’a\\
wizard cat-3S\\
\trans ‘The cat is a wizard’
\end{exe}

In example \ref{ex:intro-cat}, we see that blah blah blah…
\todo{fig:glossing abbreviations}

\mainmatter
%-----
\jichap{Phonology}\label{chap:phonology}
Phonology regards the phonemes incorporated into the language, consonants or vowels, and their use in syllables and words.

\jisec{Consonants}

\begin{table}[h]\label{tab:ipa-cons}
\begin{tabular}{|r|c|c|c|c|c|c|c|}
\hline
Manner/Point of articulation & Bilabial & Labiodental & Alveolar & Postalveolar & Palatal & Velar & Uvular\\\hline
Nasal			&   m &     &   n &     &     &     &     \\\hline
Plosive			& p b &     & t d &     &     & k g &     \\\hline
Fricative		&     & f v & s z & \ipaS \ipaZ & &&\ipaR \\\hline
Approximant		&     &     &     &     &   j &     &     \\\hline
Lateral approximant	&     &     &   l &     &     &     &     \\\hline
\end{tabular}
\caption{Consonants in Deẽreẽ}
\end{table}

Romanization is done with the IPA symbols, that is, the ones in table \ref{tab:ipa-cons}, except for \ipaS, \ipaZ~and \ipaR, which are written respectively as \emph{sh}, \emph{zh} and \emph{r}.

\jisec{Vowels}
The nine vowels are as described (IPA and orthography) in table \ref{tab:ipa-vowels}.

\begin{table}[h]\label{tab:ipa-vowels}
\begin{tabular}{|r|c|c|c|c|}
\hline
	& Front & Back \\\hline
Closed	& /i/, /y/ <ù> & /u/ \\\hline
Mid	& /e\textasciitilde\ipaE/, /\ipaET(j)/ <ẽi>, /ø{\textasciitilde}œ/ <ë> & /o\textasciitilde\ipaO/ \\\hline
Open	& /a/ & /\ipaAT/ <ã> \\\hline
\end{tabular}
\caption{Vowels in Deẽreẽ}
\end{table}

\jisec{Phonotactics}
The syllable structure is \textbf{(C|B)V(L)(F)} where:

\begin{description}
\item[C] is any consonant
\item[B] is a group of two consanants among \emph{tr, dr, pl, bl, kl, gl, kr, gr}
\item[V] is any vowel
\item[L] is a consonant among \emph{j, l, r, s}
\item[F] word-final only, a consonant among \emph{n, p, t, k, f, s, sh, l}
\end{description}



%-----
\jichap{Morphology}\label{chap:morphology}
This chapter, morphology, aims to describe \emph{what the words look like}. This means describing
the nouns, adjectives, adverbs, verbs in their basic forms. Details though will be about person
suffixes, Deẽreẽ’s equivalent for articles.

\jisec{Gender}
Deẽreẽ has three genders: Human, marked \emph{H}, Magic, marked \emph{M}, and Common, marked
\emph{C}. Every word has a gender, whatever its grammatical nature (part of speech) might be: nouns
have gender, but verbs have gender too, as well as adjectives. Words that are not nouns do not just
agree in gender (if they do), but has their own intrinsic categories.

\jissec{Formation}
To determine the gender of a word, you have to look at its last vowel (whether or not it ends in a
vowel). The rules are in table \ref{tab:morph-genders}.

\begin{table}[h]\label{tab:morph-genders}
\begin{center}
\begin{tabular}{|c|ccc|}
\hline
\textbf{Human}  & e & ẽ & u\\\hline
\textbf{Magic}  & a & o & ã\\\hline
\textbf{Common} & ü & ë & i\\\hline
\end{tabular}
\end{center}
\caption{Vowels in the Deẽreẽ genders (H:e/ẽ/u, M:a/o/ã, C:ü/ë/i)}
\end{table}

For example:
\begin{description}
\newgle{zhalbarki}{beer}\newgle{drete}{to go}\newgle{magra}{soul}
\item[\gls{zhalbarki}] beer, ending in \emph{i}, is of \emph{Common} gender
\item[\gls{drete}] to go, ending in \emph{e}, is of \emph{Human} gender
\item[\gls{magra}] soul, ending in \emph{a}, is of \emph{Magic} gender
\end{description}

\jissec{Human}
The \textbf{Human} gender regroups everything that is a non-magic person, a human made thing or
skill. Example words are: \newgle{zeẽi}{city}\gls{zeẽi}, city, \newgle{torete}{to speak}
\gls{torete}, to speak, \newgle{murtef}{door}\gls{murtef}, door, \newgle{kueste}{noble}\gls{kueste},
noble.

\jissec{Magic}
The \textbf{Magic} gender represents everything that is of magic or mystical nature. This thus
heavily depends on the culture. In the setting of Deẽreẽ, the Kingdom of Reosal, magic, shortly
explained, is the soul, present in everyone, and in many animals (\newgle{eltol}{bird}\gls{eltol}),
natural elements, and some objects as well. Nouns designating people with magical abilities, like
\newgle{fãssa}{wizard}\gls{fãssa}, wizard, are also of this gender.

Abilities of the mind (\newgle{sojmata}{to think}\gls{sojmata}, to think) also fall into this
category.

\jissec{Common}
What is \textbf{Common} is not necessarily unimportant. An example is \newgle{traütü}{to know}
\gls{traütü}, to know. However, \newgle{kretül}{mole}\gls{kretül}, mole, and a lot of other words,
are \textbf{Common}. This gender roughly regroups what doesn’t fall into the other two categories,
that is, what is not specific to humans or what isn’t magical.


\jisec{Number}
Number in Deẽreẽ is simply singular (\emph{SI}) and plural (\emph{PL}). As gender is marked with the
last vowel of a given word, number is marked with the last consonant.

\jissec{Formation}
\begin{table}[h]\label{tab:morph-number}
\begin{center}
\begin{tabular}{|c|cccccccccc|}
\hline
\textbf{Singular} & -n & -m & -p & -t & -k & -f & -s & -sh & -l & -\emph{vowel}\\\hline
\textbf{Plural}   & -n & -m & -b & -d & -g & -v & -z & -zh & -r & -r\\\hline
\end{tabular}
\end{center}
\caption{Number (singular and plural) formation}
\end{table}

For example, magra, magrar; aal; aar; \newgle{sün}{metal}\gls{sün} (metal), sün, \newgle{zërsh}{pig}
\gls{zërsh} (pig), zërzh.

\jissec{Meaning and agreement}
\jisssec{Nouns}
The meanings of singular and plural for nouns are as expected: singular means that the thing is
present once, plural that it is present several times: \newgle{zheo}{fairy}\gls{zheo}, fairy,
\gls{zheo}r, fairies.

As for uncountable nouns, singular means ‘some stuff’, while plural means ‘several kinds of the
stuff’. This also is quite an expected behavior: \newgle{loishë}{history}\gls{loishë}, history,
some part of history; \gls{loishë}r, histories, or \newgle{geni}{milk}\gls{geni}, milk, some milk;
\gls{geni}r, several kinds of milk.

\jissec{Verbs}
Verbs agree in number with the agent. See chapter Conjugation on page \pageref{chap:conjugation} for
more detail.

\jissec{Adverbs and adjectives}
These do not agree, as is explained in chapter Syntax (page \pageref{chap:syntax}).


\jisec{Person suffixes}
Person suffixes in Deẽreẽ are suffixes that can be added to different kinds of words, notably nouns,
verbs, adjectives, adverbs, or postpositions. Their purpose is to precise which person the word
refers to. The language uses six grammatical persons, the usual ones:

\begin{description}
\item[1.SI] I, speaker
\item[2.SI] singular you (thou), whom is spoken to
\item[3.SI] he/she/it, neither speaker nor person being spoken to
\item[1.PL] we, several people including the speaker
\item[2.PL] plural you, several people including the one(s) spoken to
\item[3.PL] they, several people neither speaking nor being spoken to
\end{description}

\jissec{Form and agreement}
Person suffixes have to agree with what they \emph{refer to}. This is very important because, the
word they refer to is not always the word on which they are. They decline in \emph{number},
\emph{person}, and \emph{gender}. Their basic forms, preceded by an apostrophe, are listed in table
\ref{tab:morph-basic-pers-suff}.

\begin{table}[h]
\begin{center}\label{tab:morph-basic-pers-suff}
\begin{tabular}{|c||c|c|c|c|}
\hline
\multirow{2}{*}{\textbf{Gender}} & \multicolumn{2}{|c|}{\textbf{Human}} & \multirow{2}{*}{\textbf{Magic}} & \multirow{2}{*}{\textbf{Common}}\\
                                 & \textbf{Female/neuter} & \textbf{Male} & & \\\hline\hline
\textbf{1.SI}                    & ’esh & ’ẽish & ’ash & ’ùsh \\\hline
\textbf{2.SI}                    & ’el  & ’ẽil  & ’al  & ’ùl  \\\hline
\textbf{3.SI}                    & ’e   & ’ẽi   & ’a   & ’ù   \\\hline
\textbf{1.PL}                    & ’ezh & ’ẽizh & ’azh & ’ùzh \\\hline
\textbf{2.PL}                    & ’er  & ’ẽir  & ’ar  & ’ùr  \\\hline
\textbf{3.PL}                    & ’er  & ’ẽir  & ’ar  & ’ùr  \\\hline
\end{tabular}
\end{center}
\caption{Person suffixes in their basic form}
\end{table}

We have to notice there is no difference between \textbf{2.PL} and \textbf{3.PL}. The difference is
thus done with context understanding.

\jissec{Definite}
The definite article, that is, the equivalent of \emph{the}, is formed simply by applying the person
suffix after the noun, in their basic form from table \ref{tab:morph-basic-pers-suff}. For example
see (\ref{exe:person-def}), with \newgle{mosoj}{cat}\gls{mosoj}, cat.

\begin{exe}
\ex\label{exe:person-def}
\begin{xlist}
\ex\gll mosoj’a\\
cat(MAG)-3.SI.MAG\\
\trans ‘the cat’

\ex\gll mosoj’ash\\
cat(MAG)-1.SI.MAG\\
\trans ‘I, the cat’

\ex\gll mosojr’ar\\
cat(MAG)-2.PL.MAG\\
\trans ‘you, cats’
\end{xlist}
\end{exe}

As we see in this example (\ref{exe:person-def}), the person suffixes add meaning to the words in a
short, efficient way. They are also used with verbs, as in \emph{dret’ẽil}, I go. Details on
conjugation are however in a separated chapter.

\jissec{Indefinite}
Indefinite articles express the idea that the speaker doesn’t know \emph{which} thing they are
speaking about, but \emph{a} thing. Its form in Deẽreẽ is a separate word, not a suffix. This word
is \newgle{et}{indefinite article}\emph{et}, which declines as described in table \ref{tab:morph-indef-pers-suff}.

\begin{table}[h]
\begin{center}\label{tab:morph-indef-pers-suff}
\begin{tabular}{|c||c|c|c|c|}
\hline
\multirow{2}{*}{\textbf{Gender}} & \multicolumn{2}{|c|}{\textbf{Human}} & \multirow{2}{*}{\textbf{Magic}} & \multirow{2}{*}{\textbf{Common}}\\
                                 & \textbf{Female/neuter} & \textbf{Male} & & \\\hline\hline
\textbf{1.SI}                    & etesh & etẽish & et’ash & et’ùsh \\\hline
\textbf{2.SI}                    & etel  & etẽil  & et’al  & et’ùl  \\\hline
\textbf{3.SI}                    & et    & ẽit    & eat    & eùt    \\\hline
\textbf{1.PL}                    & etezh & etẽizh & et’azh & et’ùzh \\\hline
\textbf{2.PL}                    & eter  & etẽir  & et’ar  & et’ùr  \\\hline
\textbf{3.PL}                    & ed    & ẽid    & ead    & eùd    \\\hline
\end{tabular}
\end{center}
\caption{Indefinite article \emph{et} and its declensions}
\end{table}

The position of the indefinite article is still after the noun it describes. See example
\ref{exe:person-indef}.

\begin{exe}
\ex\label{exe:person-indef}
\begin{xlist}
\ex\gll \gls{eltol} eat\\
bird(MAG) 3.SI.MAG.INDEF\\
\trans ‘a bird’

\ex\gll \newgle{shil}{worm}\gls{shil} et’ùl\\
worm(COM) 2.SI.COM.INDEF\\
\trans ‘You worm!’
\end{xlist}
\end{exe}

\jissec{Demonstrative}
Without a pretty table this time; demonstrative articles are used to designate a specific object out
of a group, as a transition from indefinite to definite. The English equivalent are \emph{this} and
\emph{that}.

These are of two types: proximal and distal.

\jisssec{Proximal demonstrative}
\jisssec{Distal demonstrative}

\jissec{Genitive}
’re
\jisssec{Owner}
todel’rẽi
\jisssec{Distinct}
ëj adal’e ãblẽi’re


\jisec{Adjectives}

Normally, before. Exceptionally, after, but just the one, and with 3SG suffix.

\jissec{Comparatives}

\begin{exe}
\ex\label{exe:comparatives-more}
\gll samas’a nal lëlü ãsta’a\\
lake-3SG through big ocean-3SG\\
\trans The ocean is bigger than the lake.
\end{exe}

\begin{exe}
\ex\label{exe:comparatives-equal}
\gll zhas’ra fil mok’a\\
water(M)-3SG.GEN transparent(C) ghost(M)-3SG\\
\trans The ghost is transparent as water.
\end{exe}





%-----
\jichap{Derivational Morphology}\label{chap:derivational-morphology}
The name ‘derivational morphology’ refers to all the mechanisms of word derivation, that is, how to
make a word out of another. These rules apply to several kinds of words, and are hereafter grouped
by part of speech they derive.

%---
\jisec{Derivations from nouns}
\jissec{Owner/Ruler: \emph{-(s)ket}}
\jisssec{Examples}
kortis (shop), kortisket (shopkeeper); lül (village), lülket (village mayor); beet (sheep), beesket (sheperd)

\jisssec{Etymology}
Comes from the verb \emph{oskete}, to own. This suffix is a reduction of the phrase \emph{<noun> osket sep}.

%---
\jisec{Derivations from adjectives}
\jissec{To turn into, to become X: \emph{-fütü}}
\jisssec{Examples}
dreer (cold), dreerfütü (to get cold); lëlü (big), lëlfütü (to grow); zaf (clean), zafütü (to clean)

\jisssec{Etymology}
From the verb \emph{fütü}, to push, or to turn into. The phrase normally used for nouns is \emph{<noun> ol fütü}.

%---
\jisec{Derivations from verbs}
\jissec{Tool: \emph{-<V>sën}}
\jisssec{Examples}
shëskata (to charm), shëskasën (charisma); ãgrete (to rule), ãgresën (authority, personality trait); kütü (to see), küsën (attention, prudence)

\jisssec{Etymology}
\todo{etymology of -sën}

%---
\jissec{Result: \emph{-ür}}
\jisssec{Examples}
dekete (to write), dekür (writings); tomete (to look like), tomür (appearance); ramata (to hurt), ramür (pain)

\jisssec{Etymology}
The postposition \emph{dür} (after), and the phrase \emph{<verb-infinitive> dür}.

%---
\jissec{Performer: \emph{-ẽi}}
\jisssec{Examples}
fete (to drink), fetẽi (drinker); bapete (to tell a lie), bapẽi (liar)

\jisssec{Etymology}
The word \emph{ẽi} (person).


%-----
\jichap{Conjugation}\label{chap:conjugation}
Verbs.


%-----
\jichap{Syntax}\label{chap:syntax}
\jisec{Noun phrase}
\emph{Deẽreẽ} is a \textbf{head-final} language. This means that in most cases if not all, the modifier words or clauses will come \emph{before} the modified item. This is true of noun-adjective order.

\begin{exe}
\label{ex:syntax-noun}
\ex
\gll lif zhok-’a\\
red blood-3S\\
\trans ‘The red blood’
\ex
\gll aso kal eat\\
good eye 3S.INDF\\
\trans ‘a good eye’
\end{exe}


\jisec{Sentence}
This section will discuss the ordering of parts of a sentence, and the morphemes needed for certain specific constructions, such as subclauses. Let us define a few notations first: \textbf{V} for Verb, \textbf{O} for the Objects of the verb, both direct and indirect, and \textbf{S} for Subject of the verb.

\jissec{Primary order}
The basic sentence word order is \textbf{OVS}.

\begin{exe}
\ex\label{ex:syntax-primary-flower}
\textbf{Ãdiir ead trasat shükẽ’e}\\
/{\ipaAT}.'dii{\ipaR} e.'ad t{\ipaR}a.'sat 'shy.k{\ipaET}.e/

\gll ãdiir ead trasat shükẽ-’e\\
flower 3S.INDF give man-3S\\
\trans ‘The man gives flowers.’

\ex\label{ex:syntax-primary-ghost}
\textbf{Mog’ar parat naiës’rel}\\
/'m{\ipaO}g.ar pa.'{\ipaR}at nai.'əs.{\ipaR\ipaE}l/

\gll mog-’ar parat naiës-’rel\\
ghost.PL-3PL toFear child-2SG.GEN\\
\trans Your child fears the ghosts.
\end{exe}

\jissec{Secondary order}
The secondary order of words, is secondary by rank. It is fairly often used, but only in particular constructs that will be treated below.

This word order is verb-first. So it can be either \textbf{VOS}, if the subject \textbf{S} is not attached to the verb in the form of a pronoun suffix. It gives \emph{V O S}.
If the subject \emph{is} a pronoun suffix on the verb, then the word order can be noted \textbf{VSO} : \emph{V’S O}.
In this latter case, the subject being attached to the verb, we can say that there is no fully described subject, and write this word order as \textbf{V(’S)O(S)}.
The chosen notation will be \textbf{VOS}.

Clauses in this order can still have adverbial modifiers come before the verb.


\jisec{Subclauses: \emph{sep/sap/süp}}
Subclauses use the secondary word order, \textbf{VOS}, and an additionnal morpheme, the particle \emph{sep/sap/süp}, depending on gender (resp. Human, Magic, Common). In the following examples, this particle will be glossed as SUBC.

Its construction will be described below, depeding on whether the subclause is nominalized or not, that is whether it acts as a noun or qualifies a noun.

\jissec{Subclause as a noun}
Subclauses may be used as nominal constructions, in which case they may include a subject, and an object if required by the verb.

We see in example \ref{ex:syntax-subclause-as-noun-1} that there is gender agreement, between the subclause particle \emph{sep} and
the verb in the clause, \emph{assete}, meaning, ‘to light’.
\begin{exe}
\ex\label{ex:syntax-subclause-as-noun-1}
\textbf{Asset milzhër’ü Nosh’a sep küt’esh.}\\
/as.'s{\ipaE}t mil.'{\ipaZ}ər.y 'no\ipaS.a s{\ipaE}p 'kyt.e\ipaS/

\gll asset milzhër-’ü Nosh-’a sep küt-’esh\\
toLight(H) forest-3S Moon-3S SUBC(H) see-1S\\
\trans ‘I see that the Moon lights the forest.’
\end{exe}

In example \ref{ex:syntax-subclause-as-noun-2}, the clause is used as a noun and followed by the benefactive postposition \emph{ol}.
Once again, there is agreement between \emph{sep} and the verb \emph{dekete}, ‘to write’.
\begin{exe}
\ex\label{ex:syntax-subclause-as-noun-2}
\textbf{Deke sep ol trasat’esh.}\\
/d{\ipaE}.'k{\ipaE} s{\ipaE}p '{\ipaO}l t{\ipaR}a.'sat.{\ipaE\ipaS}/

\gll deke sep ol trasat-’esh\\
write(H).3S SUBC(H) to give-1S\\
\trans ‘I give to the person who writes.’
\end{exe}

\jissec{Subclause as a qualifier}
Whenever a subclause qualifies a noun, the verb inside the subclause takes the qualified noun either as its subject or its object, direct or indirect.

In all three cases (subject, direct object, indirect object), what is missing in the subclause will be identified by a pronoun in the \emph{vocative} case (\emph{le/la/lü}).

This vocative pronoun will be inside the subclause (between the verb and \emph{sep/sap/süp}) in the first two cases, and directly after, on a postposition, in the case of an indirect object.\\

When the subclause is \textbf{missing subject}, a subject is placed on the verb in the subclause, as a vocative pronoun suffix.
Both this pronoun and the subclause particle agree in gender with what the subject actually is, in the case of example (\ref{ex:syntax-subclause-missing-S}),
\emph{Nosh’a}, ‘the Moon’.

\begin{exe}
\ex\label{ex:syntax-subclause-missing-S}
\textbf{Asset’la sap Nosh’a küt’esh.}\\
/as.'s{\ipaE}t.la sap 'n{\ipaO\ipaS}.a 'kyt.{\ipaE\ipaS}/

\gll asset-’la ezh sap Nosh-’a küt-’esh\\
toLight-3S(M).VOC 1PL SUBC(M) Moon(M)-3S(M) see-1S\\
\trans ‘I see the moon that lights us.’
\end{exe}

The \textbf{direct object} is missing in example \ref{ex:syntax-subclause-missing-DO}. A vocative placeholder is used within the subclause where an object
would be.
The vocative pronoun \emph{lü} as well as the subclause particle \emph{süp} agree in gender with the actual object, located after the subclause, \emph{ãdiil’ü}, ‘the flower’.

\begin{exe}
\ex\label{ex:syntax-subclause-missing-DO}
\textbf{Trasat’e lü süp ãdiil’ü küt’esh.}\\
/t{\ipaR}a.'sat.e ly syp {\ipaAT}.'diil.y 'kyt.{\ipaE\ipaS}/

\gll trasat-’e lü süp ãdiil-’ü küt-’esh\\
give-3S 3S(C).VOC SUBC(C) flower(C)-3S(C) see-1S\\
\trans ‘I see the flower s/he gives.’
\end{exe}

In the case of an indirect object missing from the subclause (example \ref{ex:syntax-subclause-missing-IO}), the agreement is still the same:
both \emph{sap} and the vocative \emph{’la} agree with the actual indirect object, \emph{sãs’a}.
However, the vocative is not within the subclause, but on the postposition \emph{after} the subclause particle.

\begin{exe}
\ex\label{ex:syntax-subclause-missing-IO}
\textbf{Aüt eltol’a sap ëi’la sãs’a küt’esh.}\\
/a.'yt {\ipaE}l.'tol.a sap 'əi.la 's{\ipaAT}s.a 'kyt.{\ipaE\ipaS}/

\gll aüt eltol-’a sap ëi-’la sãs-’a küt-’esh\\
come bird-3S SUBC(M) from-3S(M).VOC tree(M)-3S(M) see-1S\\
\trans ‘I see the tree from which the bird comes.’
\end{exe}


\jisec{Questions}
The syntax of questions requires a question particle, depending on gender:
\textbf{lu, lo, li} for \emph{Human}, \emph{Magic} and \emph{Common} respectively.
They are not to be confused with the vocative suffixes \emph{’le, ’la, ’lü}.

Questions use the \textbf{secondary word order}, (VOS), and the question particle can be inside or at the end of the sentence.

\jissec{Yes/No questions}

\begin{exe}
\ex\label{ex:syntax-yes-no-question}
\textbf{Küt’el tor’a ron eltol’a li?}\\
/'kyt.{\ipaE}l 't{\ipaO\ipaR}r.a {\ipaR\ipaO}n eltol’a li/

\gll küt-’el tor-’a ron eltol-’a li\\
see-2S mountain-3S on bird-3S Q\\
\trans ‘Do you see the bird on the mountain?’
\end{exe}

In example \ref{ex:syntax-yes-no-question}, we see that the verb is in first position,
followed by an object (\emph{tor’a ron eltol’a}, ‘the bird on the mountain’),
and the question particle \emph{li} comes in the end.\\

In a yes/no (or closed) question, the question particle is located at the end of the sentence and must agree in gender with the verb.

\jissec{Open questions}
An open question is one that would use a wh- word in English (what, who, when, etc.).
In Deẽreẽ, those are constructed with the VOS word order, and the question particle is used as (and called) a placeholder for what is asked for.
This placeholder can take qualifiers in the form of adjectives or nouns, that serve to discriminate if we ask about nature of thing, time, location, etc.
The placeholder will also take a person suffix, making it well distinct from the closed question particle.

Compare:

\begin{description}
\item[Feu dre’l le?] Are you drinking?
\item[Feu dre’l li’ü] What are you drinking?
\end{description}

\emph{le} in the first sentence is the question particle, whereas in the second sentence \emph{li} is an interrogative placeholder.
The placeholder is in the \emph{Common} gender to agree with an expected response, whose gender is yet unknown.

\begin{exe}
\ex\label{ex:syntax-open-question-1}
\textbf{Fet zhalbraki’ü lu’mer?}\\
/f{\ipaE}t {\ipaZ}al.'b{\ipaR}a.ki.y 'lu.m{\ipaE\ipaR}/

\gll fet zhalbraki-’ü lu-’mer ?\\
drink beer-3S Q(H)-2PL.DEM ?\\
\trans ‘Who of you drinks a beer?’
\end{exe}

Example \ref{ex:syntax-open-question-1} shows how the interrogative placeholder agrees with the expected answer:
its suffix is a second person demonstrative, indicating the question is asked to a group of people.





%-----
\backmatter
\clearpage\glsaddall
\begin{multicols}{2}\printglossaries\end{multicols}
\listoffigures\addcontentsline{toc}{chapter}{\listfigurename}
\listoftables\addcontentsline{toc}{chapter}{\listtablename}
\listoftodos
\end{document}

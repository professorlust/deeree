\jisec{Subclauses: \emph{sep/sap/süp}}
Subclauses use the secondary word order, \textbf{VOS}, and an additionnal morpheme, the particle \emph{sep/sap/süp}, depending on gender (resp. Human, Magic, Common). In the following examples, this particle will be glossed as SUBC.

Its construction will be described below, depeding on whether the subclause is nominalized or not, that is whether it acts as a noun or qualifies a noun.

\jissec{Subclause as a noun}
Subclauses may be used as nominal constructions, in which case they may include a subject, and an object if required by the verb.

We see in example \ref{ex:syntax-subclause-as-noun-1} that there is gender agreement, between the subclause particle \emph{sep} and
the verb in the clause, \emph{assete}, meaning, ‘to light’.
\begin{exe}
\ex\label{ex:syntax-subclause-as-noun-1}
\textbf{Asset milzhër’ü Nosh’a sep küt’esh.}\\
/as.'s{\ipaE}t mil.'{\ipaZ}ər.y 'no\ipaS.a s{\ipaE}p 'kyt.e\ipaS/

\gll asset milzhër-’ü Nosh-’a sep küt-’esh\\
toLight(H) forest-3S Moon-3S SUBC(H) see-1S\\
\trans ‘I see that the Moon lights the forest.’
\end{exe}

In example \ref{ex:syntax-subclause-as-noun-2}, the clause is used as a noun and followed by the benefactive postposition \emph{ol}.
Once again, there is agreement between \emph{sep} and the verb \emph{dekete}, ‘to write’.
\begin{exe}
\ex\label{ex:syntax-subclause-as-noun-2}
\textbf{Deke sep ol trasat’esh.}\\
/d{\ipaE}.'k{\ipaE} s{\ipaE}p '{\ipaO}l t{\ipaR}a.'sat.{\ipaE\ipaS}/

\gll deke sep ol trasat-’esh\\
write(H).3S SUBC(H) to give-1S\\
\trans ‘I give to the person who writes.’
\end{exe}

\jissec{Subclause as a qualifier}
Whenever a subclause qualifies a noun, the verb inside the subclause takes the qualified noun either as its subject or its object, direct or indirect.

In all three cases (subject, direct object, indirect object), what is missing in the subclause will be identified by a pronoun in the \emph{vocative} case (\emph{le/la/lü}).

This vocative pronoun will be inside the subclause (between the verb and \emph{sep/sap/süp}) in the first two cases, and directly after, on a postposition, in the case of an indirect object.\\

When the subclause is \textbf{missing subject}, a subject is placed on the verb in the subclause, as a vocative pronoun suffix.
Both this pronoun and the subclause particle agree in gender with what the subject actually is, in the case of example (\ref{ex:syntax-subclause-missing-S}),
\emph{Nosh’a}, ‘the Moon’.

\begin{exe}
\ex\label{ex:syntax-subclause-missing-S}
\textbf{Asset’la sap Nosh’a küt’esh.}\\
/as.'s{\ipaE}t.la sap 'n{\ipaO\ipaS}.a 'kyt.{\ipaE\ipaS}/

\gll asset-’la ezh sap Nosh-’a küt-’esh\\
toLight-3S(M).VOC 1PL SUBC(M) Moon(M)-3S(M) see-1S\\
\trans ‘I see the moon that lights us.’
\end{exe}

The \textbf{direct object} is missing in example \ref{ex:syntax-subclause-missing-DO}. A vocative placeholder is used within the subclause where an object
would be.
The vocative pronoun \emph{lü} as well as the subclause particle \emph{süp} agree in gender with the actual object, located after the subclause, \emph{ãdiil’ü}, ‘the flower’.

\begin{exe}
\ex\label{ex:syntax-subclause-missing-DO}
\textbf{Trasat’e lü süp ãdiil’ü küt’esh.}\\
/t{\ipaR}a.'sat.e ly syp {\ipaAT}.'diil.y 'kyt.{\ipaE\ipaS}/

\gll trasat-’e lü süp ãdiil-’ü küt-’esh\\
give-3S 3S(C).VOC SUBC(C) flower(C)-3S(C) see-1S\\
\trans ‘I see the flower s/he gives.’
\end{exe}

In the case of an indirect object missing from the subclause (example \ref{ex:syntax-subclause-missing-IO}), the agreement is still the same:
both \emph{sap} and the vocative \emph{’la} agree with the actual indirect object, \emph{sãs’a}.
However, the vocative is not within the subclause, but on the postposition \emph{after} the subclause particle.

\begin{exe}
\ex\label{ex:syntax-subclause-missing-IO}
\textbf{Aüt eltol’a sap ëi’la sãs’a küt’esh.}\\
/a.'yt {\ipaE}l.'tol.a sap 'əi.la 's{\ipaAT}s.a 'kyt.{\ipaE\ipaS}/

\gll aüt eltol-’a sap ëi-’la sãs-’a küt-’esh\\
come bird-3S SUBC(M) from-3S(M).VOC tree(M)-3S(M) see-1S\\
\trans ‘I see the tree from which the bird comes.’
\end{exe}


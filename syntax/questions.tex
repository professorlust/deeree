\jisec{Questions}
The syntax of questions requires a question particle, depending on gender:
\textbf{lu, lo, li} for \emph{Human}, \emph{Magic} and \emph{Common} respectively.
They are not to be confused with the vocative suffixes \emph{’le, ’la, ’lü}.

Questions use the \textbf{secondary word order}, (VOS), and the question particle can be inside or at the end of the sentence.

\jissec{Yes/No questions}

\begin{exe}
\ex\label{ex:syntax-yes-no-question}
\textbf{Küt’el tor’a ron eltol’a li?}\\
/'kyt.{\ipaE}l 't{\ipaO\ipaR}r.a {\ipaR\ipaO}n eltol’a li/

\gll küt-’el tor-’a ron eltol-’a li\\
see-2S mountain-3S on bird-3S Q\\
\trans ‘Do you see the bird on the mountain?’
\end{exe}

In example \ref{ex:syntax-yes-no-question}, we see that the verb is in first position,
followed by an object (\emph{tor’a ron eltol’a}, ‘the bird on the mountain’),
and the question particle \emph{li} comes in the end.\\

In a yes/no (or closed) question, the question particle is located at the end of the sentence and must agree in gender with the verb.

\jissec{Open questions}
An open question is one that would use a wh- word in English (what, who, when, etc.).
In Deẽreẽ, those are constructed with the VOS word order, and the question particle is used as (and called) a placeholder for what is asked for.
This placeholder can take qualifiers in the form of adjectives or nouns, that serve to discriminate if we ask about nature of thing, time, location, etc.
The placeholder will also take a person suffix, making it well distinct from the closed question particle.

Compare:

\begin{description}
\item[Feu dre’l le?] Are you drinking?
\item[Feu dre’l li’ü] What are you drinking?
\end{description}

\emph{le} in the first sentence is the question particle, whereas in the second sentence \emph{li} is an interrogative placeholder.
The placeholder is in the \emph{Common} gender to agree with an expected response, whose gender is yet unknown.

\begin{exe}
\ex\label{ex:syntax-open-question-1}
\textbf{Fet zhalbraki’ü lu’mer?}\\
/f{\ipaE}t {\ipaZ}al.'b{\ipaR}a.ki.y 'lu.m{\ipaE\ipaR}/

\gll fet zhalbraki-’ü lu-’mer ?\\
drink beer-3S Q(H)-2PL.DEM ?\\
\trans ‘Who of you drinks a beer?’
\end{exe}

Example \ref{ex:syntax-open-question-1} shows how the interrogative placeholder agrees with the expected answer:
its suffix is a second person demonstrative, indicating the question is asked to a group of people.


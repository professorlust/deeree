\jisec{Sentence}
This section will discuss the ordering of parts of a sentence, and the morphemes needed for certain specific constructions, such as subclauses. Let us define a few notations first: \textbf{V} for Verb, \textbf{O} for the Objects of the verb, both direct and indirect, and \textbf{S} for Subject of the verb.

\jissec{Primary order}
The basic sentence word order is \textbf{OVS}.

\begin{exe}
\ex\label{ex:syntax-primary-flower}
\textbf{Ãdiir ead trasat shükẽ’e}\\
/{\ipaAT}.'dii{\ipaR} e.'ad t{\ipaR}a.'sat 'shy.k{\ipaET}.e/

\gll ãdiir ead trasat shükẽ-’e\\
flower 3S.INDF give man-3S\\
\trans ‘The man gives flowers.’

\ex\label{ex:syntax-primary-ghost}
\textbf{Mog’ar parat naiës’rel}\\
/'m{\ipaO}g.ar pa.'{\ipaR}at nai.'əs.{\ipaR\ipaE}l/

\gll mog-’ar parat naiës-’rel\\
ghost.PL-3PL toFear child-2SG.GEN\\
\trans Your child fears the ghosts.
\end{exe}

\jissec{Secondary order}
The secondary order of words, is secondary by rank. It is fairly often used, but only in particular constructs that will be treated below.

This word order is verb-first. So it can be either \textbf{VOS}, if the subject \textbf{S} is not attached to the verb in the form of a pronoun suffix. It gives \emph{V O S}.
If the subject \emph{is} a pronoun suffix on the verb, then the word order can be noted \textbf{VSO} : \emph{V’S O}.
In this latter case, the subject being attached to the verb, we can say that there is no fully described subject, and write this word order as \textbf{V(’S)O(S)}.
The chosen notation will be \textbf{VOS}.

Clauses in this order can still have adverbial modifiers come before the verb.


Phonology regards the phonemes incorporated into the language, consonants or vowels, and their use in syllables and words.

\jisec{Consonants}

\begin{table}[h]\label{tab:ipa-cons}
\begin{tabular}{|r|c|c|c|c|c|c|c|}
\hline
Manner/Point of articulation & Bilabial & Labiodental & Alveolar & Postalveolar & Palatal & Velar & Uvular\\\hline
Nasal			&   m &     &   n &     &     &     &     \\\hline
Plosive			& p b &     & t d &     &     & k g &     \\\hline
Fricative		&     & f v & s z & \ipaS \ipaZ & &&\ipaR \\\hline
Approximant		&     &     &     &     &   j &     &     \\\hline
Lateral approximant	&     &     &   l &     &     &     &     \\\hline
\end{tabular}
\caption{Consonants in Deẽreẽ}
\end{table}

Romanization is done with the IPA symbols, that is, the ones in table \ref{tab:ipa-cons}, except for \ipaS, \ipaZ~and \ipaR, which are written respectively as \emph{sh}, \emph{zh} and \emph{r}.

\jisec{Vowels}
The nine vowels are as described (IPA and orthography) in table \ref{tab:ipa-vowels}.

\begin{table}[h]\label{tab:ipa-vowels}
\begin{tabular}{|r|c|c|c|c|}
\hline
	& Front & Back \\\hline
Closed	& /i/, /y/ <ù> & /u/ \\\hline
Mid	& /e\textasciitilde\ipaE/, /\ipaET(j)/ <ẽi>, /ø{\textasciitilde}œ/ <ë> & /o\textasciitilde\ipaO/ \\\hline
Open	& /a/ & /\ipaAT/ <ã> \\\hline
\end{tabular}
\caption{Vowels in Deẽreẽ}
\end{table}

\jisec{Phonotactics}
The syllable structure is \textbf{(C|B)V(L)(F)} where:

\begin{description}
\item[C] is any consonant
\item[B] is a group of two consanants among \emph{tr, dr, pl, bl, kl, gl, kr, gr}
\item[V] is any vowel
\item[L] is a consonant among \emph{j, l, r, s}
\item[F] word-final only, a consonant among \emph{n, p, t, k, f, s, sh, l}
\end{description}


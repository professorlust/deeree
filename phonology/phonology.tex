Phonology regards the phonemes incorporated into the language, consonants or vowels, and their use in syllables and words.

\jisec{Consonants}

\begin{table}[h]\label{tab:phon-ipa-cons}
\begin{center}
\begin{tabular}{|r|c|c|c|c|c|c|c|}
\hline
Manner/Point of articulation & Bilabial & Labiodental & Alveolar & Postalveolar & Palatal & Velar & Uvular\\\hline
Nasal			&   m &     &   n &     &     &     &     \\\hline
Plosive			& p b &     & t d &     &     & k g &     \\\hline
Fricative		&     & f v & s z & \ipaS \ipaZ & &&\ipaR \\\hline
Approximant		&     &     &     &     &   j &     &     \\\hline
Lateral approximant	&     &     &   l &     &     &     &     \\\hline
\end{tabular}
\end{center}
\caption{Consonants in Deẽreẽ}
\end{table}

Romanization is done with the IPA symbols, that is, the ones in table \ref{tab:phon-ipa-cons}, except for \ipaS, \ipaZ~and \ipaR, which are written respectively as \emph{sh}, \emph{zh} and \emph{r}.

\jissec{Consonant pairs}
The combinations of consonants allowed are the following: \textbf{tr, dr, pl, bl, kl, gl, kr, gr}. Other will occur due to phonotactics, but no other beginning in a plosive.

\jisec{Vowels}
The nine vowels are as described (IPA and orthography) in table \ref{tab:phon-ipa-vowels}.

\begin{table}[h]\label{tab:phon-ipa-vowels}
\begin{center}
\begin{tabular}{|r|c|c|c|c|}
\hline
	& Front & Back \\\hline
Closed	& /i/, /y/ <ù> & /u/ \\\hline
Mid	& /e\textasciitilde\ipaE/, /\ipaET(j)/ <ẽi>, /ø{\textasciitilde}œ/ <ë> & /o\textasciitilde\ipaO/ \\\hline
Open	& /a/ & /\ipaAT/ <ã> \\\hline
\end{tabular}
\end{center}
\caption{Vowels in Deẽreẽ}
\end{table}

\jissec{On /{\ipaET}j/ or /\ipaET/}
One of the doubtful phonemes in Deẽreẽ’s phonology is the grapheme <ẽi>’s pronunciation. It can be pronounced with palatalization or not, that is, /{\ipaET}j/ or /\ipaET/. How to pronounce it is normally determined by its position in the word.

As the last phoneme of a word <ẽi> will be palatalized: /{\ipaET}j/. But when <ẽi> is initial or inside a word, it will be pronounced as either palatalized or not. When followed by j, l, s, n, p, t, or sh, that is a consonant from bilabial to palatal but not further back (k, g, r), it is palatalized, otherwise it is not.

\jisec{Phonotactics}
The syllable structure is \textbf{(C|B)V(L)(F)} where:
\begin{description}
\item[C] is any consonant
\item[B] is a group of two consonants among \emph{tr, dr, pl, bl, kl, gl, kr, gr}
\item[V] is any vowel
\item[L] is a consonant among \emph{j, l, r, s}
\item[F] word-final only, a consonant among \emph{n, p, t, k, f, s, sh, l}
\end{description}

\jissec{Vowel pairs}
Quite often, two vowels will appear in a word consecutively. In this case, there are three cases: long vowel, diaeresis, or semi-vowel.

\jisssec{Long vowel sounds}
In a word like \newgle{aal}{wind}\gls{aal}, There is a long vowel sound due to vowel reduplication. There, it is pronounced as [a.al], instead of [a:l]. That is, the real pronounciation of this word is as close to diaeresis as possible, as opposed to a ‘real’ long vowel.

\jisssec{Diaeresis}
\newgle{tëfuete}{to wipe}
\newgle{fùã}{powder}
Two consecutive vowels will give a diaeresis, that is, the two vowels being pronounced separately without any semi-vowel, in words like \gls{fùã}, powder, or before a verb ending, for example in \gls{tëfuete}, to wipe.

\jisssec{Semi-vowel}
\newgle{marùëlf}{nature}
A semi-vowel, that is, a sound like /j/, /w/, or /ɥ/ (the \emph{u} in the French huit), is used in various words, for example the Deẽreẽ word for nature, \gls{marùëlf}.


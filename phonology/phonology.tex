Phonology regards the phonemes incorporated into the language, consonants or vowels, and their use
in syllables and words.

\jisec{Consonants}

\begin{table}
\begin{center}
\begin{tabular}{|r|c|c|c|c|c|c|}
\hline
Point of articulation & Bilabial & Labiodental & Alveolar & Postalveolar & Velar & Uvular\\\hline
Nasal			      &   m &     &   n &                 &     &       \\\hline
Plosive               & p b &     & t d &                 & k g &       \\\hline
Fricative             &     & f v & s z & {\ipaS} {\ipaZ} &     & \ipaR \\\hline
Lateral approximant   &     &     &   l &                 &     &       \\\hline
\end{tabular}
\end{center}
\caption{Consonants in Deẽreẽ}
\label{tab:phon-ipa-cons}
\end{table}

\jissec{Romanization}
Romanization is done with the IPA symbols, that is, the ones in table \ref{tab:phon-ipa-cons},
except for \ipaS, \ipaZ~and \ipaR, which are written respectively as \emph{sh}, \emph{zh} and
\emph{r}. Pronunciation of these three letters is:

\begin{description}
\item[sh] as in English ‘\emph{sh}eep’
\item[zh] as in English ‘vi\emph{si}on’
\item[r]  as in French  ‘F\emph{r}ance’
\end{description}

\jissec{Consonant pairs}
The combinations of consonants allowed are the following: \textbf{pr, br, tr, dr, kr, gr}.
Other will occur due to phonotactics, but no other beginning in a plosive.

\jisec{Vowels}
The nine vowels are as described (IPA and orthography) in table \ref{tab:phon-ipa-vowels}.

\begin{table}
\begin{center}
\begin{tabular}{|r|c|c|c|c|}
\hline
	& Front & Back \\\hline
Closed	& /i/, /y/ & /u/ \\\hline
Mid	& /e/, /\ipaET/, /ə/ & /o/ \\\hline
Open	& /a/ & /\ipaAT/ \\\hline
\end{tabular}
\end{center}
\caption{Vowels in Deẽreẽ}
\label{tab:phon-ipa-vowels}
\end{table}

\jissec{Romanization}
As for consonants, we use IPA symbols, except for the a few exceptions. See table \ref{tab:phon-vowels-orthography}.

\begin{table}
\begin{center}
\begin{tabular}{|r|c|c|c|c|c|c|c|c|c|}
\hline
IPA          & /i/ & /y/ & /u/ & /e/ & /\ipaET/ & /ə/ & /o/ & /a/ & /\ipaAT/\\\hline
Romanization & <i> & <ü> & <u> & <e> & <ẽ>      & <ë> & <o> & <a> & <ã>     \\\hline
\end{tabular}
\end{center}
\caption{Vowels orthography}
\label{tab:phon-vowels-orthography}
\end{table}

\jissec{Allowed diphthongs}
Allowed diphthongs are only the following: /ii, ai, {\ipaET}i, əi, oi/.

\jisec{Syllables}
The syllable structure is given below (onset-nucleus-coda):

\begin{description}
\item[onset] can be any single consonant, or \emph{pr, br, tr, dr, kr, gr};
\item[nucleus] can be any vowel or diphthong;
\item[coda] can be \emph{<l>, <r> or <s>}, but only if nucleus is not a diphthong.
\end{description}

\jisec{Stress}
Stress in words is on the \textbf{last syllable} when the word ends in a \emph{consonant or diphthong}, and on the \textbf{second-to-last syllable} when the word ends in a \emph{single vowel}.

For example:
\begin{description}
\item[zhal-bra-ki] [{\ipaZ}al.'b{\ipaR}a.ki] (beer)
\item[a-shëi] [a.'{\ipaS}ëi] (temple)
\item[lë-lü] ['lə.ly] (big)
\item[tra-sa-ta] [t{\ipaR}a.'sa.ta] (to give)
\end{description}


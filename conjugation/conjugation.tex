Here I’ll explain the different tense/aspect grams in \emph{Deẽreẽ}. We’ll use three example verbs in this chapter: \emph{dekete}, ‘to write’; \emph{trasata}, ‘to give’, and \emph{aütü}, ‘to come’.

\jisec{Simple tenses}
To construct the present simple, one has to apply a pronoun suffix to the verb root (which ends in \emph{et/at/üt}).
However if the verb’s gender is the same as the pronoun’s gender and this suffix is indefinite,
the verb’s final \emph{t} and the suffix’s vowel may be ommited.
See table \ref{tab:conj-present-simple}.

\begin{table}[h]
\begin{center}
\begin{tabular}{|c|c|c|c|}\hline

\textbf{Verb}   & dekete (H) & trasata (M) & aütü (C) \\\hline
\textbf{Human}  & deke’sh, deke’l, deke & trasat’esh, trasat’el, trasat’e & aüt’esh, aüt’el, aüt’e \\\hline
\textbf{Magic}  & deket’ash, deket’al, deket’a & trasa’sh, trasa’l, trasa & aüt’ash, aüt’al, aüt’a \\\hline
\textbf{Common} & deket’üsh, deket’ül, deket’ü & trasat’üsh, trasat’ül, trasat’ü & aü’sh, aü’l, aü \\\hline

\end{tabular}
\end{center}
\caption{Examples of \emph{dekete}, \emph{trasata} and \emph{aütü} in the present simple tense, 1S, 2S \& 3S}
\label{tab:conj-present-simple}
\end{table}

The past simple is constructed in a manner similar to the present simple, except that here the verb endings \emph{et/at/üt}
are replaced with \emph{ẽi/oi/ëi}.
See table \ref{tab:conj-past-simple}.

\begin{table}[h]
\begin{center}
\begin{tabular}{|c|c|c|c|}\hline

\textbf{Verb}   & dekete (H) & trasata (M) & aütü (C) \\\hline
\textbf{Human}  & dekẽi’sh, dekẽi’l, dekẽi & trasoi’esh, trasoi’el, trasoi’e & aëi’esh, aëi’el, aëi’e \\\hline
\textbf{Magic}  & dekẽi’ash, dekẽi’al, dekẽi’a & traoi’sh, trasoi’l, trasoi & aëi’ash, aëi’al, aëi’a \\\hline
\textbf{Common} & dekẽi’üsh, dekẽi’ül, dekẽi’ü & trasoi’üsh, trasoi’ül, trasoi’ü & aëi’sh, aëi’l, aëi \\\hline

\end{tabular}
\end{center}
\caption{Examples of \emph{dekete}, \emph{trasata} and \emph{aütü} in the past simple tense, 1S, 2S \& 3S}
\label{tab:conj-past-simple}
\end{table}

\jisec{Complex tenses}
The previous tenses were simple tenses. The tenses in this section will require a \emph{verb participle} to be constructed.

The verb participle is constructed by replacing the infinitive ending with either \emph{eu, aã} or \emph{üi}, depending on gender. Note that those are \emph{not} diphthongs, and should not be pronounced as such.\\

Continuing our example:
\begin{description}
\item[dekete] \emph{dekeu} /de.'ke.u/
\item[trasata] \emph{trasaã} /t{\ipaR}a.'sa.ã/
\item[aütü] \emph{aüi} /a.'y.i/
\end{description}

The complex tenses are constructed using the verb’s participle followed by a conjuged auxilary verb, either \emph{sete}, ‘to do’, or \emph{drete}, ‘to go’.
See table \ref{tab:conj-complex-tenses} for reference.

\begin{table}[h]
\begin{center}
\begin{tabular}{|c|c|c|}\hline

\textbf{Tense of auxilary} & \textbf{Present} & \textbf{Past} \\\hline
PTCP + sete & present progressive & past progressive \\\hline
PTCP + drete & future & past inchoative \\\hline

\end{tabular}
\end{center}
\caption{Complex tenses depending on auxilary verb and tense}
\label{tab:conj-complex-tenses}
\end{table}

Examples of the various tenses:
\begin{description}
\item[aüi se’a] it(M) is coming (present progressive)
\item[dekeu sẽi’sh] I(H) was writing (past progressive)
\item[aüi dre’sh] I(H) will come (future)
\item[trasaã drẽi’l] you(H) started to give (past inchoative)
\end{description}


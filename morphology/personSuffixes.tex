\jisec{Person suffixes}
Person suffixes in Deẽreẽ are suffixes that can be added to different kinds of words, notably nouns,
verbs, adjectives, adverbs, or postpositions. Their purpose is to precise which person the word
refers to. The language uses six grammatical persons, the usual ones:

\begin{description}
\item[1.SI] I, speaker
\item[2.SI] singular you (thou), whom is spoken to
\item[3.SI] he/she/it, neither speaker nor person being spoken to
\item[1.PL] we, several people including the speaker
\item[2.PL] plural you, several people including the one(s) spoken to
\item[3.PL] they, several people neither speaking nor being spoken to
\end{description}

\jissec{Form and agreement}
Person suffixes have to agree with what they \emph{refer to}. This is very important because, the
word they refer to is not always the word on which they are. They decline in \emph{number},
\emph{person}, and \emph{gender}. Their basic forms, preceded by an apostrophe, are listed in table
\ref{tab:morph-basic-pers-suff}.

\begin{table}[h]
\begin{center}\label{tab:morph-basic-pers-suff}
\begin{tabular}{|c||c|c|c|c|}
\hline
\multirow{2}{*}{\textbf{Gender}} & \multicolumn{2}{|c|}{\textbf{Human}} & \multirow{2}{*}{\textbf{Magic}} & \multirow{2}{*}{\textbf{Common}}\\
                                 & \textbf{Female/neuter} & \textbf{Male} & & \\\hline\hline
\textbf{1.SI}                    & ’esh & ’ẽish & ’ash & ’ùsh \\\hline
\textbf{2.SI}                    & ’el  & ’ẽil  & ’al  & ’ùl  \\\hline
\textbf{3.SI}                    & ’e   & ’ẽi   & ’a   & ’ù   \\\hline
\textbf{1.PL}                    & ’ezh & ’ẽizh & ’azh & ’ùzh \\\hline
\textbf{2.PL}                    & ’er  & ’ẽir  & ’ar  & ’ùr  \\\hline
\textbf{3.PL}                    & ’er  & ’ẽir  & ’ar  & ’ùr  \\\hline
\end{tabular}
\end{center}
\caption{Person suffixes in their basic form}
\end{table}

We have to notice there is no difference between \textbf{2.PL} and \textbf{3.PL}. The difference is
thus done with context understanding.

\jissec{Definite}
The definite article, that is, the equivalent of \emph{the}, is formed simply by applying the person
suffix after the noun, in their basic form from table \ref{tab:morph-basic-pers-suff}. For example
see (\ref{exe:person-def}), with \newgle{mosoj}{cat}\gls{mosoj}, cat.

\begin{exe}
\ex\label{exe:person-def}
\begin{xlist}
\ex\gll mosoj’a\\
cat(MAG)-3.SI.MAG\\
\trans ‘the cat’

\ex\gll mosoj’ash\\
cat(MAG)-1.SI.MAG\\
\trans ‘I, the cat’

\ex\gll mosojr’ar\\
cat(MAG)-2.PL.MAG\\
\trans ‘you, cats’
\end{xlist}
\end{exe}

As we see in this example (\ref{exe:person-def}), the person suffixes add meaning to the words in a
short, efficient way. They are also used with verbs, as in \emph{dret’ẽil}, I go. Details on
conjugation are however in a separated chapter.

\jissec{Indefinite}
Indefinite articles express the idea that the speaker doesn’t know \emph{which} thing they are
speaking about, but \emph{a} thing. Its form in Deẽreẽ is a separate word, not a suffix. This word
is \newgle{et}{indefinite article}\emph{et}, which declines as described in table \ref{tab:morph-indef-pers-suff}.

\begin{table}[h]
\begin{center}\label{tab:morph-indef-pers-suff}
\begin{tabular}{|c||c|c|c|c|}
\hline
\multirow{2}{*}{\textbf{Gender}} & \multicolumn{2}{|c|}{\textbf{Human}} & \multirow{2}{*}{\textbf{Magic}} & \multirow{2}{*}{\textbf{Common}}\\
                                 & \textbf{Female/neuter} & \textbf{Male} & & \\\hline\hline
\textbf{1.SI}                    & etesh & etẽish & et’ash & et’ùsh \\\hline
\textbf{2.SI}                    & etel  & etẽil  & et’al  & et’ùl  \\\hline
\textbf{3.SI}                    & et    & ẽit    & eat    & eùt    \\\hline
\textbf{1.PL}                    & etezh & etẽizh & et’azh & et’ùzh \\\hline
\textbf{2.PL}                    & eter  & etẽir  & et’ar  & et’ùr  \\\hline
\textbf{3.PL}                    & ed    & ẽid    & ead    & eùd    \\\hline
\end{tabular}
\end{center}
\caption{Indefinite article \emph{et} and its declensions}
\end{table}

The position of the indefinite article is still after the noun it describes. See example
\ref{exe:person-indef}.

\begin{exe}
\ex\label{exe:person-indef}
\begin{xlist}
\ex\gll \gls{eltol} eat\\
bird(MAG) 3.SI.MAG.INDEF\\
\trans ‘a bird’

\ex\gll \newgle{shil}{worm}\gls{shil} et’ùl\\
worm(COM) 2.SI.COM.INDEF\\
\trans ‘You worm!’
\end{xlist}
\end{exe}

\jissec{Demonstrative}
Without a pretty table this time; demonstrative articles are used to designate a specific object out
of a group, as a transition from indefinite to definite. The English equivalent are \emph{this} and
\emph{that}.

These are of two types: proximal and distal.

\jisssec{Proximal demonstrative}
\jisssec{Distal demonstrative}

\jissec{Genitive}
’re
\jisssec{Owner}
todel’rẽi
\jisssec{Distinct}
ëj adal’e ãblẽi’re


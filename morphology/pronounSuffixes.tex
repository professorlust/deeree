\jisec{Pronoun suffixes}
Pronoun suffixes in Deẽreẽ are suffixes that can be added to different kinds of words, notably nouns,
verbs, adjectives, adverbs, or postpositions. Their purpose is to precise which person the word
refers to. The language uses six grammatical persons, the usual ones:

\begin{description}
\item[1.SI] I, speaker
\item[2.SI] singular you (thou), whom is spoken to
\item[3.SI] he/she/it, neither speaker nor person being spoken to
\item[1.PL] we, several people including the speaker
\item[2.PL] plural you, several people including the one(s) spoken to
\item[3.PL] they, several people neither speaking nor being spoken to
\end{description}

\jissec{Form and agreement}
Pronoun suffixes have to agree with what they \emph{refer to}. This is very important because, the
word they refer to is not always the word on which they are. They decline in \emph{number},
\emph{person}, and \emph{gender}. Their basic forms, preceded by an apostrophe, are listed in table
\ref{tab:morph-basic-pers-suff}.

\begin{table}[h]
\begin{center}
\begin{tabular}{|c||c|c|c|}
\hline
\textbf{Gender} & \textbf{Human} & \textbf{Magic} & \textbf{Common}\\\hline
\textbf{1.SI}                    & ’esh & ’ash & ’üsh \\\hline
\textbf{2.SI}                    & ’el  & ’al  & ’ül  \\\hline
\textbf{3.SI}                    & ’e   & ’a   & ’ü   \\\hline
\textbf{1.PL}                    & ’ezh & ’azh & ’üzh \\\hline
\textbf{2.PL}                    & ’er  & ’ar  & ’ür  \\\hline
\textbf{3.PL}                    & ’er  & ’ar  & ’ür  \\\hline
\end{tabular}
\end{center}
\caption{Pronoun suffixes in their basic form}
\label{tab:morph-basic-pers-suff}
\end{table}

We have to notice there is no difference between \textbf{2.PL} and \textbf{3.PL}. The difference is
thus done with context understanding.

\jissec{Definite}
The definite article, that is, the equivalent of \emph{the}, is formed simply by applying the person
suffix after the noun, in their basic form from table \ref{tab:morph-basic-pers-suff}. For example
see (\ref{exe:person-def}), with \newgle{mosoi}{cat}\gls{mosoi}, cat.

\begin{exe}
\ex\label{exe:person-def}
\begin{xlist}
\ex\textbf{Mosoi’a}\\
/mo.'s{\ipaO}i.a/

\gll mosoi-’a\\
cat(MAG)-3.SI.MAG\\
\trans ‘The cat’

\ex\textbf{Mosoi’ash}\\
/mo.'s{\ipaO}i.a/

\gll mosoi-’ash\\
cat(MAG)-1.SI.MAG\\
\trans ‘I, the cat’

\ex\textbf{Mosoir’ar}\\
/mo.'s{\ipaO}i{\ipaR}.a{\ipaR}/

\gll mosoir’ar\\
cat(MAG)-2.PL.MAG\\
\trans ‘You cats’
\end{xlist}
\end{exe}

As we see in this example (\ref{exe:person-def}), the pronoun suffixes add meaning to the words in a
short, efficient way. They are also used with verbs, as in \emph{dret’ẽil}, I go. Details on
conjugation are however in a separated chapter.

\jissec{Indefinite}
Indefinite articles express the idea that the speaker doesn’t know \emph{which} thing they are
speaking about, but \emph{a} thing. Its form in Deẽreẽ is a separate word, not a suffix. This word
is \newgle{et}{indefinite article}\emph{et}, which declines as described in table \ref{tab:morph-indef-pers-suff}.

\begin{table}[h]
\begin{center}\label{tab:morph-indef-pers-suff}
\begin{tabular}{|c||c|c|c|}
\hline
\textbf{Gender} & \textbf{Human} & \textbf{Magic} & \textbf{Common}\\\hline
\textbf{1.SI}                    & etesh & et’ash & et’üsh \\\hline
\textbf{2.SI}                    & etel  & et’al  & et’ül  \\\hline
\textbf{3.SI}                    & et    & eat    & eüt    \\\hline
\textbf{1.PL}                    & etezh & et’azh & et’üzh \\\hline
\textbf{2.PL}                    & eter  & et’ar  & et’ür  \\\hline
\textbf{3.PL}                    & ed    & ead    & eüd    \\\hline
\end{tabular}
\end{center}
\caption{Indefinite article \emph{et} and its declensions}
\end{table}

The position of the indefinite article is still after the noun it describes. See example
\ref{exe:person-indef}.

\begin{exe}
\ex\label{exe:person-indef}
\begin{xlist}
\ex\textbf{Eltol eat}\\
/{\ipaE}l.'t{\ipaO}l e.'at/

\gll \gls{eltol} eat\\
bird(MAG) 3.SI.MAG.INDEF\\
\trans ‘a bird’

\ex\textbf{Shil et’ül!}\\
/{\ipaS}il '{\ipaE}t.yl/

\gll \newgle{shil}{worm}\gls{shil} et-’ül\\
worm(COM) INDEF-2.SI.COM\\
\trans ‘You’re a worm!’
\end{xlist}
\end{exe}

\jissec{Demonstrative}
Without a pretty table this time; demonstrative articles are used to designate a specific object out
of a group, as a transition from indefinite to definite. The English equivalent are \emph{this} and
\emph{that}.

These are of two types: proximal and distal.

\jisssec{Proximal demonstrative}
The word proximal means \emph{close}, so a proximal demonstrative is an article designating a thing
close to the speaker (\emph{this}). In Deẽreẽ, it is formed with the definite suffixes preceded with
the letter \emph{m}.

\begin{exe}
\ex\label{exe:person-prox-dem}
\textbf{Mok’ma}\\
/'m{\ipaO}k.ma/

\gll mok-’ma\\
ghost(MAG)-3.SI.DEM\\
\trans ‘This ghost’
\end{exe}

\jisssec{Distal demonstrative}
As proximal means close, distal means \emph{remote}. It is so the equivalent of \emph{that}. It is
made in the language with the definite suffixes preceded with \emph{asm}. It is the occasion here to
introduce the word \newgle{senü}{fish}\gls{senü}, fish.

\begin{exe}
\ex\label{exe:person-dist-dem}
\textbf{Senü’asmü}\\
/'se.ny.as.my/

\gll \gls{senü}-’asmü\\
fish(COM)-3.SI.ADEM\\
\trans ‘That fish’
\end{exe}

\jissec{Genitive}
It is useful to repeat that the pronoun suffix on a word does \emph{not} necessarily refer to this
word, and agrees with what it refers to. In this part, we’ll describe how pronoun suffixes can
express the notion of property, or a general link between two objects.

Its shape is very simple: prefix the suffix with the letter \emph{R} (\emph{’e} becomes \emph{’re}).
It is always used to refer to the \emph{owner} in a property relationship.

\jisssec{Overt owner}
This is like a genitive word case. A word with a \emph{’re} suffix is the owner of the following
word; it works like an adjective.

\newgle{adal}{old}\newgle{brẽi}{bread}
\begin{exe}
\ex\label{exe:person-gen-2obj}
\textbf{Adal’rel brẽi’e.}\\
/a.'dal.{\ipaR\ipaE}l 'b{\ipaR\ipaET}.e/

\gll \gls{adal}-’rel, \gls{brẽi}-’e\\
old-2.SI.GEN bread-3.SI\\
\trans ‘Your bread, old man.’
\end{exe}

\jisssec{Non-overt owner}
Another case of use exists, where the owner is not explicitely given. A genitive personal suffix can
be used on a word alone, not being used as a qualifier. This means the object is owned by the person
the suffix agrees with.

\newgle{todel}{mother}
\begin{exe}
\ex\label{exe:person-gen-1obj}
\textbf{Todel’resh}\\
/to.'d{\ipaE}l.r{\ipaE\ipaS}

\gll \gls{todel}-’resh\\
mother-1.SI.GEN\\
\trans ‘My mother’
\end{exe}

\jissec{Vocative}
The vocative is used to call out to someone, to state existence of something, as well as in subclauses (see chapter Syntax on page \pageref{chap:syntax}).
It is constructed by adding an \textbf{L} in front of the suffix.

\begin{exe}
\ex\label{ex:person-voc-1}
\textbf{Eltol’la!}\\
/{\ipaE}l.'t{\ipaO}l.la/

\gll eltol-’la\\
bird-3S.VOC\\
\trans ‘There is a bird!’
\end{exe}


\jisec{Numerals}
This section discusses how to count in Deẽreẽ. The language uses a \emph{base 12} counting system.

\jissec{Cardinal numbers}
The terms for numbers 1 to 11 are given in the table below:

\begin{center}\begin{tabular}{|l|c|c|c|c|c|c|c|c|c|c|c|}\hline

\textbf{Number} & 1 & 2 & 3 & 4 & 5 & 6 & 7 & 8 & 9 & 10 & 11 \\\hline
\textbf{Deẽreẽ} & ki & miki & as & toni & taki & mias & mibe & tol & kani & üma & moi\\\hline

\end{tabular}\end{center}

Powers of 12 and 0 have other meanings in the language:

\begin{center}\begin{tabular}{|l|c|c|c|c|c|}\hline

\textbf{Number} & $0$ & $12$ & $12^2 = 144$ & $12^3 = 1728$ & $12^4 = 20736$ \\\hline
\textbf{Deẽreẽ} & dal & shëi & lül & zeẽ & kënak \\\hline
\textbf{Meaning} & nothing & house & village & city & palace \\\hline

\end{tabular}\end{center}

Numbers are constructed with the conjunction \emph{ak}, ‘and’, and the word \emph{kon}, ‘number, quantity’. For example:

\begin{description}
\item[$42 = 3 \times 12 + 6$] as shëi ak mias kon
\item[$200 = 144 + 4 \times 12 + 8$] lül ak toni shëi ak tol kon
\end{description}

\jissec{Ordinal numbers}
\todo{ordinals}

\jissec{Partital numbers}
\todo{fractions}

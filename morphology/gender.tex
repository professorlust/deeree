\jisec{Gender}
Deẽreẽ has three genders: Human, marked \emph{H}, Magic, marked \emph{M}, and Common, marked
\emph{C}. Every word has a gender, whatever its grammatical nature (part of speech) might be: nouns
have gender, but verbs have gender too, as well as adjectives. Words that are not nouns do not just
agree in gender (if they do), but have their own intrinsic categories.

\jissec{Formation}
To determine the gender of a word, you have to look at its last vowel (whether or not it ends in a
vowel). The rules are in table \ref{tab:morph-genders}.

\begin{table}[h]
\begin{center}
\begin{tabular}{|c|ccc|}
\hline
\textbf{Human}  & e & ẽ & u\\\hline
\textbf{Magic}  & a & o & ã\\\hline
\textbf{Common} & ü & ë & i\\\hline
\end{tabular}
\end{center}
\caption{Vowels in the Deẽreẽ genders (H:e/ẽ/u, M:a/o/ã, C:ü/ë/i)}
\label{tab:morph-genders}
\end{table}

Diphthongs put the word in the gender determined by the first diphthong vowel.
This means words ending in \emph{-ẽi} are Human, those ending in \emph{-ai/-oi} are Magic, and those ending in \emph{-ëi/-ii} are Common.

For example:
\begin{description}
\newgle{zhalbraki}{beer}\newgle{drete}{to go}\newgle{magra}{soul}\newgle{shëi}{house}
\item[\gls{zhalbraki}] beer, ending in \emph{i}, is of \emph{Common} gender
\item[\gls{drete}] to go, ending in \emph{e}, is of \emph{Human} gender
\item[\gls{magra}] soul, ending in \emph{a}, is of \emph{Magic} gender
\item[\gls{shëi}] house, ending in \emph{ëi}, is of \emph{Common} gender
\end{description}

\jissec{Human}
The \textbf{Human} gender regroups everything that is a non-magic person, a human made thing or
skill. Example words are: \newgle{zeẽ}{city}\gls{zeẽ}, city, \newgle{torete}{to speak}
\gls{torete}, to speak, \newgle{murtef}{door}\gls{murtef}, door, \newgle{kueste}{noble}\gls{kueste},
noble.

\jissec{Magic}
The \textbf{Magic} gender represents everything that is of magic or mystical nature. This thus
heavily depends on the culture. In the setting of Deẽreẽ, the Kingdom of Reosal, magic, shortly
explained, is the soul, present in everyone, and in many animals (\newgle{eltol}{bird}\gls{eltol}, bird),
natural elements, and some objects as well. Nouns designating people with magical abilities, like
\newgle{fãssa}{wizard}\gls{fãssa}, wizard, are also of this gender.

Abilities of the mind (\newgle{soimata}{to think}\gls{soimata}, to think) also fall into this
category.

\jissec{Common}
What is \textbf{Common} is not necessarily unimportant. An example is \newgle{toütü}{to know}
\gls{toütü}, to know. However, \newgle{kretül}{mole}\gls{kretül}, mole, and a lot of other words,
are \textbf{Common}. This gender roughly regroups what doesn’t fall into the other two categories,
that is, what is not specific to humans or what isn’t magical.


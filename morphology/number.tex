\jisec{Number}
Number in Deẽreẽ is simply singular (\emph{SI}) and plural (\emph{PL}). As gender is marked with the
last vowel of a given word, number is marked with the last consonant.

\jissec{Formation}
\begin{table}[h]\label{tab:morph-number}
\begin{center}
\begin{tabular}{|c|ccccccccc|}
\hline
\textbf{Singular} & -l & -n & -p & -t & -k & -f & -s & -sh & -\emph{vowel}\\\hline
\textbf{Plural}   & -r & -m & -b & -d & -g & -v & -z & -zh & -r\\\hline
\end{tabular}
\end{center}
\caption{Number (singular and plural) formation}
\end{table}

For example, magra, magrar; aal; aar; \newgle{sùn}{metal}\gls{sùn} (metal), sùm, \newgle{zërsh}{pig}
\gls{zërsh} (pig), zërzh.

\jissec{Meaning and agreement}
\jisssec{Nouns}
The meanings of singular and plural for nouns are as expected: singular means that the thing is
present once, plural that it is present several times: \newgle{zheo}{fairy}\gls{zheo}, fairy,
\gls{zheo}r, fairies.

As for uncountable nouns, singular means ‘some stuff’, while plural means ‘several kinds of the
stuff’. This also is quite an expected behavior: \newgle{lojshë}{history}\gls{lojshë}, history,
some part of history; \gls{lojshë}r, histories, or \newgle{geni}{milk}\gls{geni}, milk, some milk;
\gls{geni}r, several kinds of milk.

\jissec{Verbs}
Verbs agree in number with the agent. See chapter Conjugation on page \pageref{chap:conjugation} for
more detail.

\jissec{Adverbs and adjectives}
These do not agree, as is explained in chapter Syntax (page \pageref{chap:syntax}).


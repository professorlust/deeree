\jisec{Adjectives}

They come before the noun, and do not agree with it in any way.

\jissec{Comparatives and Superlatives}
Comparatives are ways of expressing that something is *more*, *less* or *as much X as* something else.

The comparative in Deẽreẽ is placed before the comparee, as a qualifier.
It is constructed in three parts: \textbf{reference-scale-direction}.

\begin{description}
\item[The reference] followed by \emph{nal}, ‘through’, is what the comparee will be compared to.
\item[The scale] states on what criteria the comparison will be.
\item[The direction] indicates if the comparee is more, less or equal to the reference in the given scale.
\end{description}

\begin{table}[h]
\begin{center}
\begin{tabular}{|llll|l|}\hline

\textbf{Reference} & \textbf{Scale} & \textbf{Direction} & \textbf{Comparee} & \textbf{Translation} \\\hline

shükẽ’me \emph{nal} & lëlü & & azhe’me & this woman is \textbf{taller} than this man \\\hline
samis eat \emph{nal} & dreri & \textbf{rar} & edan eat & a river is \textbf{less} wide than a lake \\\hline
ãda’a \emph{nal} & lif & \textbf{sho} & zhok’a & the blood is \textbf{equally as} red as the rose \\\hline

\end{tabular}
\end{center}
\caption{Comparatives}
\label{tab:morphology-comparatives}
\end{table}

Table \ref{tab:morphology-comparatives} shows through examples how the comparative is constructed. Vocabulary used in this table:

\begin{description}
\item[shükẽ] man (male)
\item[azhe] woman
\item[lëlü] big, tall
\item[samis] lake
\item[edan] river
\item[dreri] wide
\item[ãda] rose (flower)
\item[zhok] blood
\item[lif] red
\end{description}

Table \ref{tab:morphology-comparatives} shows well the role of the postposition \emph{nal}:
its meaning is, ‘relative to’, ‘by reference to’.
Hence a sentence like \emph{shükẽ’me nal lëlü azhe’me} can be litterally translated as:
‘Relative to this man, this woman is tall.’, meaning ‘This woman is taller than this man.’ (See vocabulary list above).

A superlative is constructed the same way as a comparative, except that we compare with everything: to be the \emph{best} means to be \emph{better than everything}.

\begin{exe}
\ex\label{ex:morphology-superlatives}

\textbf{Un’ur nal mama aso ẽi’esh sep are’sh!}\\
/'un.u{\ipaR} nal 'ma.ma 'a.so '{\ipaET}i.e{\ipaS} s{\ipaE}p a.'{\ipaR\ipaE\ipaS}/

\gll un-’ur nal mama aso esh sep are-’sh\\
all-3PL CMP very good 1S SUBC want-1S\\
\trans ‘I wanna be the very best!’

\end{exe}

The superlative of example \ref{ex:morphology-superlatives} litterally means, ‘Relative to everything, me be very good, I want’.


The name ‘derivational morphology’ refers to all the mechanisms of word derivation, that is, how to
make a word out of another. These rules apply to several kinds of words, and are hereafter grouped
by part of speech they derive.

\jisec{Derivations from nouns}
See table \ref{tab:noun-derivations}.

\begin{table}[h]
\begin{center}
\begin{tabular}{|l|l|c|l|}\hline

\textbf{Construction} & \textbf{Part of speech} & \textbf{Short} & \textbf{Description} \\\hline

\end{tabular}
\end{center}
\caption{Noun derivations}
\label{tab:noun-derivations}
\end{table}

\jisec{Derivations from verbs}
See table \ref{tab:verb-derivations}.

\begin{table}[h]
\begin{center}
\begin{tabular}{|l|l|c|l|}\hline

\textbf{Construction} & \textbf{Part of speech} & \textbf{Short} & \textbf{Description} \\\hline

-ete/-ata/-ütü => -esën/-asën/-üsën & noun & tool & that with which action is done \\\hline

\end{tabular}
\end{center}
\caption{Verb derivations}
\label{tab:verb-derivations}
\end{table}

\todo{1t1 --> 31n: result, that which X does\\torete: to say --> toruen: speech}


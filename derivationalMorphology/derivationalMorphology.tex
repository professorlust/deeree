The name ‘derivational morphology’ refers to all the mechanisms of word derivation, that is, how to
make a word out of another. These rules apply to several kinds of words, and are hereafter grouped
by part of speech they derive.

%---
\jisec{Derivations from nouns}
\jissec{Owner/Ruler: \emph{-(s)ket}}
\jisssec{Examples}
kortis (shop), kortisket (shopkeeper); lül (village), lülket (village mayor); beet (sheep), beesket (sheperd)

\jisssec{Etymology}
Comes from the verb \emph{oskete}, to own. This suffix is a reduction of the phrase \emph{<noun> osket sep}.

%---
\jisec{Derivations from adjectives}
\jissec{To turn into, to become X: \emph{-fütü}}
\jisssec{Examples}
dreer (cold), dreerfütü (to get cold); lëlü (big), lëlfütü (to grow); zaf (clean), zafütü (to clean)

\jisssec{Etymology}
From the verb \emph{fütü}, to push, or to turn into. The phrase normally used for nouns is \emph{<noun> ol fütü}.

%---
\jisec{Derivations from verbs}
\jissec{Tool: \emph{-<V>sën}}
\jisssec{Examples}
shëskata (to charm), shëskasën (charisma); ãgrete (to rule), ãgresën (authority, personality trait); kütü (to see), küsën (attention, prudence)

\jisssec{Etymology}
\todo{etymology of -sën}

%---
\jissec{Result: \emph{-ür}}
\jisssec{Examples}
dekete (to write), dekür (writings); tomete (to look like), tomür (appearance); ramata (to hurt), ramür (pain)

\jisssec{Etymology}
The postposition \emph{dür} (after), and the phrase \emph{<verb-infinitive> dür}.

%---
\jissec{Performer: \emph{-ẽi}}
\jisssec{Examples}
fete (to drink), fetẽi (drinker); bapete (to tell a lie), bapẽi (liar)

\jisssec{Etymology}
The word \emph{ẽi} (person).

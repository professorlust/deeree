The name ‘derivational morphology’ refers to all the mechanisms of word derivation, that is, how to
make a word out of another. These rules apply to several kinds of words, and are hereafter grouped
by part of speech they derive.

\jisec{Noun derivations}
\jisec{Verb derivations}
\todo{1t1 --> 2lf: tool, that with which X is done\\trasata: to give --> trasãlf: generosity}
\todo{1t1 --> 31n: result, that which X does\\torete: to say --> toruen: speech}

